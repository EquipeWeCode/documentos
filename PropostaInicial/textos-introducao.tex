
% ----------------------------------------------------------
% Introdução
% ----------------------------------------------------------
\chapter[Introdução]{Introdução}

Nesse capítulo serão mostrados os principais pontos do nosso projeto, os objetivos e quais os problemas que queremos solucionar com nossa aplicação.

\section{Justificativa}
%O problema encontrado
Existe, na contemporaneidade, uma grande dificuldade em adquirir experiência profissional através da prática de estágio, muitas vezes obrigatória no projeto pedagógico de curso das universidades. Tal problema se dá por meio das plataformas que disponibilizam tais vagas, porém com uma certa cobrança injusta em relação a habilidades que o candidato precisa possuir previamente. É também notável que existe uma certa dificuldade de conexão entre a empresa e o candidato, que muitas vezes não obtém o retorno sobre o processo de seleção da vaga.

\section{Proposta de solução}
%Descrição geral da nossa proposta
Tendo em vista os problemas anteriormente descritos, o Portal de Vagas de Estágio é um sistema para aproximar novos estudantes e empresas com vagas de estágio disponíveis, de modo que os candidatos possam receber indicações de vagas condizentes com seu perfil e empresas recebam recomendações de candidatos possivelmente adequados às vagas anunciadas.

\section{Objetivos}
O objetivo principal da nossa solução é promover um meio de conexão mais direto entre os estudantes em busca de estágio e empresas que buscam interessados em suas vagas de estágio alinhados com o perfil buscado. Através do sistema de recomendações, tantos os estudantes quanto as empresas têm papel ativo no processo de encontrar um(a) estudante/vaga ideal, cujas as competências e perfil sejam condizentes com o que é procurado.

A partir do nosso objetivo principal, podemos listar alguns objetivos mais práticos da nossa solução:

\begin{itemize}
	\item Realizar o gerenciamento de vagas entre os candidatos e as empresas de uma forma simplificada;
	\item Recomendar vagas para estudantes, empresas para estudantes, estudantes para vagas/empresas;
	\item Manter um histórico de vagas aplicadas pelo estudante;
	\item Manter um histórico de candidatos aplicados a vaga;
	\item Exibir uma linha do tempo da situação da vaga;
	\item Alertar os estudantes aplicados à vaga sobre cada mudança em seu status;
	\item Possibilitar o gerenciamento da vaga pela empresa que a registrou/publicou;
	\item Possibilitar que a empresa possas acionar (entrar em contato) com os estudantes recomendados/aplicados à vaga;
	\item Possibilitar que a empresa realize mudanças no status da vaga;
\end{itemize}
	
