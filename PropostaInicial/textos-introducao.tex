
% ----------------------------------------------------------
% Introdução
% ----------------------------------------------------------
\chapter[Introdução]{Introdução}

%Colocar algum texto aqui, não deixar em branco

\section{Justificativa}
%O problema encontrado

\section{Proposta de solução}
%Descrição geral da nossa proposta
O Portal de vagas de estágio é um sistema para aproximar novos profissionais/estudantes da área de TI e empresas com vagas de estágio/trainee disponíveis, de modo que os candidatos possam receber indicações de vagas condizentes com seu perfil e empresas recebam recomendações de candidatos possivelmente adequados às vagas anunciadas.

\section{Objetivos}
O objetivo principal da nossa solução é promover um meio de conexão mais direto entre os estudantes em busca de estágio e empresas que buscam interessados em suas vagas de estágio alinhados com o perfil buscado. Através do sistema de recomendações, tantos os estudantes quanto as empresas têm papel ativo no processo de encontrar um(a) estudante/vaga ideal, cujas as competências e perfil sejam condizentes com o que é procurado.

A partir do nosso objetivo principal, podemos listar alguns objetivos mais práticos da nossa solução:
%Eu acho que esses itens são os requisitos funcionais, não?
\begin{itemize}
	\item Realizar o gerenciamento de vagas entre os candidatos e as empresas de uma forma simplificada;
	\item Recomendar vagas para estudantes, empresas para estudantes, estudantes para vagas/empresas;
	\item Manter um histórico de vagas aplicadas pelo estudante;
	\item Manter um histórico de candidatos aplicados a vaga;
	\item Exibir uma linha do tempo da situação da vaga;
	\item Alertar os estudantes aplicados à vaga sobre cada mudança em seu status;
	\item Possibilitar o gerenciamento da vaga pela empresa que a registrou/publicou;
	\item Possibilitar que a empresa possas acionar (entrar em contato) com os estudantes recomendados/aplicados à vaga;
	\item Possibilitar que a empresa realize mudanças no status da vaga;
\end{itemize}
	
