%% Adaptado a partir de :
%%    abtex2-modelo-trabalho-academico.tex, v-1.9.2 laurocesar
%% para ser um modelo para os trabalhos no IFSP-SPO

\documentclass[
    % -- opções da classe memoir --
    12pt,               % tamanho da fonte
    openright,          % capítulos começam em pág ímpar (insere página vazia caso preciso)
    %twoside,            % para impressão em verso e anverso. Oposto a oneside
    oneside,
    a4paper,            % tamanho do papel. 
    % -- opções da classe abntex2 --
    %chapter=TITLE,     % títulos de capítulos convertidos em letras maiúsculas
    %section=TITLE,     % títulos de seções convertidos em letras maiúsculas
    %subsection=TITLE,  % títulos de subseções convertidos em letras maiúsculas
    %subsubsection=TITLE,% títulos de subsubseções convertidos em letras maiúsculas
    % Opções que não devem ser utilizadas na versão final do documento
    %draft,              % para compilar mais rápido, remover na versão final
    %MODELO,             % indica que é um documento modelo então precisa dos geradores de texto
    %TODO,                indica que deve apresentar lista de pendencias 
    % -- opções do pacote babel --
    english,            % idioma adicional para hifenização
    brazil              % o último idioma é o principal do documento
    ]{ifsp-spo-inf-ctds}

        
% ---

% --- 
% CONFIGURAÇÕES DE PACOTES
% --- 
%\usepackage{etoolbox}
%\patchcmd{\thebibliography}{\chapter*}{\section*}{}{}
\usepackage{float}
\usepackage{quoting} %para citações com várias linhas
\usepackage{pdfpages}%para cincluir pdfs \includepdf{<arquivo.pdf>}
\usepackage{pspicture}
\usepackage{qrcode}
% ---
% Informações de dados para CAPA e FOLHA DE ROSTO
% ---
\titulo{EstagiEI\\
	\emph{Website} de vagas de estágio}

% Trabalho individual
%\autor{JOSÉ BRAZ DE ARAUJO}

% Trabalho em Equipe
% ver também https://github.com/abntex/abntex2/wiki/FAQ#como-adicionar-mais-de-um-autor-ao-meu-projeto
\renewcommand{\imprimirautor}{
\begin{tabular}{lr}
Bruna da Silva Pires & SP3056651 \\
Daniel Roberto Pereira & SP3046702 \\
Igor Nathan de Oliveira Rocha & SP305263X \\
Leonardo Marques da Silva & SP3052591 \\
Lucas Lima de Santana & SP3046559 \\
Marcelo Carlos Olimpio Junior &SP3046583 \\
\end{tabular}
}


\tipotrabalho{Projeto da Disciplina PI1A5}

\disciplina{PI1A5 - Projeto Integrado I}

\preambulo{Documentação de Mínimo Produto Viável para aprovação na disciplina de Projeto Integrado I no 1º semestre de 2022.}

\data{2022}

% Definir o que for necessário e comentar o que não for necessário
% Utilizar o Nome Completo, abntex tem orientador e coorientador
% então vão ser utilizados na definição de professor
\renewcommand{\orientadorname}{Professor:}
\orientador{Carlos Henrique Veríssimo Pereira}
%\renewcommand{\coorientadorname}{Professor:}
%\coorientador{NOME COMPLETO DO PROFESSOR2}



% ---



% ---
% Configurações de aparência do PDF final


% informações do PDF
\makeatletter
\hypersetup{
        %pagebackref=true,
        pdftitle={\@title}, 
        pdfauthor={\@author},
        pdfsubject={\imprimirpreambulo},
        pdfcreator={LaTeX with abnTeX2},
        pdfkeywords={abnt}{latex}{abntex}{abntex2}{trabalho acadêmico}, 
        colorlinks=true,            % false: boxed links; true: colored links
        linkcolor=blue,             % color of internal links
        citecolor=blue,             % color of links to bibliography
        filecolor=magenta,              % color of file links
        urlcolor=blue,
        bookmarksdepth=4
}
\makeatother
% --- 

% ---

% ----
% Início do documento
% ----
\begin{document}

% Retira espaço extra obsoleto entre as frases.
\frenchspacing 

\pretextual

% ---
% Capa
% ---
\imprimircapa

% ---
% Folha de rosto
% (o * indica que haverá a ficha bibliográfica)
% ---
\imprimirfolhaderosto
%\imprimirfolhaderosto*
% ---
\newpage


% ---
% inserir lista de ilustrações
% ---
\pdfbookmark[0]{\listfigurename}{lof}
\listoffigures*
\cleardoublepage
% ---

% ---
% inserir lista de quadros
% ---
\pdfbookmark[0]{\listofquadrosname}{loq}
\listofquadros*
\cleardoublepage
% ---

\input{pre-siglas}
% ---
% inserir o sumario
% ---
\pdfbookmark[0]{\contentsname}{toc}
\tableofcontents*
\cleardoublepage
% ---
% ----------------------------------------------------------
% ELEMENTOS TEXTUAIS
% ----------------------------------------------------------
\textual
% ----------------------------------------------------------
% INTRODUÇÃO
% ----------------------------------------------------------
% ----------------------------------------------------------
% PLANEJAMENTO E GERENCIAMENTO DO PROJETO
% ----------------------------------------------------------
\chapter{Planejamento e Gerenciamento do Projeto}
Neste capítulo abordaremos a metodologia e ferramenta da gestão da equipe e do projeto, os papéis dos integrantes da equipe e informações a cerca do cronograma sendo seguido no desenvolvimento do projeto e sua documentação.
% ----------------------------------------------------------
% METODOLOGIA DE GESTÃO
% ----------------------------------------------------------
% ----------------------------------------------------------
% GESTÃO E DESENVOLVIMENTO DO PROJETO
% ----------------------------------------------------------
\section{Gestão e Desenvolvimento do Projeto}
A equipe decidiu por utilizar a metodologia ágil \textit{\gls{scrum}}, juntamente com a ferramenta de gerenciamento \gls{jira}. O \textit{\gls{scrum}} possui três fases, uma inicial de planejamento geral, uma intermediária de produção e uma final de encerramento. A fase intermediária se trata de uma série de iterações, onde em cada iteração são desenvolvidas atividades/funcionalidades a serem entregues/incrementadas. Estas iterações são chamadas de \textit{\glspl{sprint}} \cite{sommerville}, cuja duração é fixa e a equipe decidiu por durar uma semana (7 dias) no primeiro semestre e duas semanas (14 dias) no segundo semestre. Todas as atividades, elementos e artefatos que precisaram ser produzidos foram organizados, monitorados e atribuídos aos membros da equipe via \gls{jira}, onde foi possível verificar o status da atividade, assim como marcar prazos.
% ----------------------------------------------------------
% ORGANIZAÇÃO DA EQUIPE
% ----------------------------------------------------------
% ----------------------------------------------------------
% ORGANIZAÇÃO DA EQUIPE
% ----------------------------------------------------------
\section{Organização da equipe}
Após avaliarmos as principais competências de cada integrante da equipe, resolvemos separar as tarefas de cada um como indicado no \autoref{responsabilidades}.

\begin{quadro}[H]
	\caption{Divisão de responsabilidades da equipe.}
	\centering
	\begin{tabular}{| p{0.30\linewidth} | c | c | c | c | c | c | c |}
			\hline
			\thead[l]{Responsabilidade} & \thead{Bruna} & \thead{Daniel} & \thead{Igor} & \thead{Leonardo} & \thead{Lucas} & \thead{Marcelo}\\
			\hline
			Back-End. &  &  & X & X &  & X\\
			\hline
			Front-End. & X & X &  & X & X & \\
			\hline
			Banco de Dados. &  & X & X &  &  & \\
			\hline
			Blog. & X & X & X & X & X & X\\
			\hline
			Documentação. & X & X & X & X & X & X\\
			\hline
			Design. & X &  &  &  & X & \\
			\hline
			Gestão. & X &  &  &  &  & \\
			\hline
			
		\end{tabular}
	\fonte{Os Autores}
	\label{responsabilidades}
\end{quadro}

Considerando os papéis inerentes ao \gls{scrum} e as responsabilidades expostas no \autoref{responsabilidades}, o papel do \gls{scrummaster} será desempenhado pela integrante Bruna da Silva Pires, já a equipe de desenvolvimento será composta por todos os integrantes da equipe, sem exceção.

%doc da tabela: https://docs.google.com/document/d/1fCR9h-WhtyQsylmvLcXadEjcmHYoxb3MX57AzWBtPfs/edit

% ----------------------------------------------------------
% GESTÃO DE TEMPO / CRONOGRAMA
% ----------------------------------------------------------
% ----------------------------------------------------------
% GESTÃO DE TEMPO / CRONOGRAMA
% ----------------------------------------------------------
\section{Cronograma}

No início do projeto tínhamos uma organização dos macro itens (\textit{Epics}) que precisavam ser desenvolvidos com base nas entregas da disciplina de PI1A5 e dentro dos \textit{Epics} estipulamos as tarefas a serem feitas. Por meio do \gls{jira} podemos ter uma visão geral do andamento dos \textit{Epics} e os prazos, além das \textit{\glspl{sprint}} planejadas, realizadas e aquela que está em andamento.

\begin{figure}[H]
	\centering
	\caption{\label{jira-geral}Roteiro Geral de PI1A5}
	\includegraphics[width=0.95\textwidth]{../imagens/cronograma-geral.png}
	\fonte{Os Autores}
\end{figure}

A \autoref{jira-geral} mostra à esquerda a lista dos \textit{Epics} considerados para a construção do sistema \emph{EstagiEI} no início do projeto, começando pelo Desenho da Aplicação, então Prova de Conceito, \ac{mvp}, Testes, Documentação da Entrega Final, Correções e Melhorias e Blog. Na parte superior estão as datas e o período englobado por cada \textit{\gls{sprint}}. As marcações em azul mostram o período de duração de cada \textit{Epic}. A seguir, \autoref{jira-detalhe1} e \autoref{jira-detalhe2} mostram de modo mais próximo a figura anterior para fins de melhor visualização.

\begin{figure}[H]
	\centering
	\caption{\label{jira-detalhe1}Roteiro Geral de PI1A5 - Detalhe Inicial}
	\includegraphics[width=0.95\textwidth]{../imagens/cronograma-detalhe1.png}
	\fonte{Os Autores}
\end{figure}

\begin{figure}[H]
	\centering
	\caption{\label{jira-detalhe2}Roteiro Geral de PI1A5 - Detalhe Final}
	\includegraphics[width=0.95\textwidth]{../imagens/cronograma-detalhe2.png}
	\fonte{Os Autores}
\end{figure}

Apresentamos em \autoref{cronogramasem1} as \textit{\glspl{sprint}} e algumas informações expostas em \autoref{jira-geral}, \autoref{jira-detalhe1} e \autoref{jira-detalhe2} do que foi planejado e realizado em PI1A5.

\begin{quadro}[H]
	\caption{Cronograma de \glspl{sprint} - 1º semestre}
	\centering
	\begin{tabular}{| p{0.17\linewidth}  | c | c | p{0.25\linewidth} | c |}
		\hline
		\thead[l]{Sprint} & \thead{Data Inicial} & \thead{Data Final} & \thead[l]{Descrição} & \thead{Status}\\
		\hline
		Desenho da aplicação 1 & 18/04/22 & 25/04/22 & Elaboração da documentação do Desenho da Aplicação. & Concluída\\
		\hline
		Desenho da aplicação 2 & 25/04/22 & 02/05/22 &  Continuação da elaboração do Desenho da Aplicação. Planejamento para a \ac{poc}. & Concluída\\
		\hline
		POC & 02/05/22 & 09/05/22 & Finalização do Desenho da Aplicação. Início do desenvolvimento dos itens da \ac{poc} & Concluída \\
		\hline
		POC 2 & 09/05/22 & 16/05/22 & Continuação do desenvolvimento dos itens da \ac{poc}. & Concluída\\
		\hline
		MVP & 16/05/22 & 23/05/22 & Aproveitamento do que foi desenvolvido para a \ac{poc} com melhorias e ampliação conforme possível para o \ac{mvp}. & Concluída\\
		\hline
		MVP 2 & 23/05/22 & 30/05/22 & Continuação do trabalho no desenvolvimento do \ac{mvp}. & Concluída\\
		\hline
		Revisões: código e documentos & 30/05/22 & 06/06/22 &  Finalização e revisão tanto do desenvolvimento quanto da documentação. & Concluída\\
		\hline
		Preparação para a Apresentação & 06/06/22 & 13/06/22 &  Organização e planejamento da apresentação do projeto e sua documentação. & Concluída\\
		\hline
		Ajustes finais & 13/06/22 & 20/06/22 &  Ajustes a serem feitos para correção e/ou melhoria do projeto apresentado. & Concluída\\
		\hline
		Ajustes finais 2 & 20/06/22 & 27/06/22 &  Continuação de correções e ajustes para a entrega do projeto no semestre. & Concluída\\
		\hline
		Ajustes finais 3 & 27/06/22 & 04/07/22 &  Finalização dos ajustes finais e correções para a entrega definitiva do projeto no semestre. & Concluída\\
		\hline
		
	\end{tabular}
	\fonte{Os Autores}
	\label{cronogramasem1}
\end{quadro}

Para a continuação do projeto, percebemos que uma mudança de planejamento seria necessária. Assim, modificamos nossos \textit{Epics} a fim de estarem mais alinhados com o projeto em desenvolvimento, possuindo subitens de Funcionalidades que se referem a partes menores do produto em si, as quais por sua vez contém as Histórias de Usuário, que descrevem as ações/funções de cada usuário dentro so sistema.

Como o \gls{jira} não possui uma divisão intermediária entre \textit{Epics} e Histórias de Usuário, a seguir apresentamos alguns esquemas da visão que temos das divisões do projeto:

\begin{figure}[H]
	\centering
	\caption{\label{epic-empresa}\textit{Epic} da Empresa}
	\includegraphics[width=0.95\textwidth]{../imagens/epic-empresa.png}
	\fonte{Os Autores}
\end{figure}

Em \autoref{epic-empresa} está a Área da Empresa, que seria uma grande fatia do projeto, contendo as Funcionalidades pertinentes a entidade Empresa e o que se relaciona com ela, como o Gerenciamento de Vagas, o qual se abre em diversas ações que são as Histórias de Usuário dessa funcionalidade.

\begin{figure}[H]
	\centering
	\caption{\label{epic-estudante}\textit{Epic} do Estudante}
	\includegraphics[width=0.95\textwidth]{../imagens/epic-estudante.png}
	\fonte{Os Autores}
\end{figure}

Do mesmo modo como o \textit{Epic} da Empresa, o \textit{Epic} do Estudante também possui Funcionalidades pertinentes a entidade Estudante e as ações que um usuário deste tipo precisa ter no sistema, como o Gerenciamento de Candidaturas.

Além de reestruturarmos o modo como vemos e trabalhamos as etapas do projeto, também adaptamos as \textit{\glspl{sprint}} para serem de duas semanas ao invés de uma, devido à situação de tempo dos membros da equipe. O \autoref{jira-roteiro-geral-sem2} ilustra a nova organização de \textit{Epics} dividas entre \gls{backend} e \gls{frontend} e as \textit{\glspl{sprint}} de suas semanas.

\begin{figure}[H]
	\centering
	\caption{\label{jira-roteiro-geral-sem2}Roteiro Geral para PI2A6}
	\includegraphics[width=0.95\textwidth]{../imagens/jira-roteiro-geral-sem2.png}
	\fonte{Os Autores}
\end{figure}

Considerando tudo o que já foi apontado, nosso Cronograma (a seguir) se tornou mais genérico, pois a cada \gls{sprint} definimos o que seria feito, portanto apenas há algo definido explicitamente no início e no fim da segunda etapa de desenvolvimento, como a entrega final.

\begin{quadro}[H]
	\caption{Cronograma de \glspl{sprint} - 2º semestre}
	\centering
	\begin{tabular}{| p{1.5cm}  | c | c | p{4.5cm} | c |}
		\hline
		\thead[l]{Sprint} & \thead{Data Inicial} & \thead{Data Final} & \thead[l]{Descrição} & \thead{Status}\\
		\hline
		\textit{Sprint} 1 & 11/08/22 & 25/08/22 & Reorganização da equipe, reestruturação do código e do projeto de modo geral & Concluída\\
		\hline
		\textit{Sprint} 2 & 25/08/22 & 08/09/22 &  Refinamento, ajustes e adaptações do que já foi produzido; adequação da documentação existente às alterações & Concluída\\
		\hline
		\textit{Sprint} 3 & 08/09/22 & 22/09/22 & Desenvolvimento e testes de novas funcionalidades do sistema & Concluída \\
		\hline
		\textit{Sprint} 4 & 22/09/22 & 06/10/22 & Desenvolvimento e testes de novas funcionalidades do sistema & Concluída\\
		\hline
		\textit{Sprint} 5 & 06/10/22 & 20/10/22 & Desenvolvimento e testes de novas funcionalidades do sistema & Concluída\\
		\hline
		\textit{Sprint} 6 & 20/10/22 & 03/11/22 & Desenvolvimento e testes de novas funcionalidades do sistema & Concluída\\
		\hline
		\textit{Sprint} 7 & 03/11/22 & 17/11/22 & Ajustes finais, Entrega e Apresentação da aplicação & Em Progresso\\
		\hline
		\textit{Sprint} 8 & 17/11/22 & 01/12/22 & Correções e ajustes; Entrega final & Não Iniciada\\
		\hline
		
	\end{tabular}
	\fonte{Os Autores}
	\label{cronogramasem2}
\end{quadro}
% ----------------------------------------------------------
% ESTATÍSTICA DOS REPOSITÓRIOS
% ----------------------------------------------------------
% ----------------------------------------------------------
% ESTATÍSTICA DOS REPOSITÓRIOS
% ----------------------------------------------------------
\section{Estatísticas dos repositórios}

\subsection{SVN}
%gls\{svn}


\subsection{GitHub}

% ----------------------------------------------------------
% <CAPÍTULO 2>
% ----------------------------------------------------------
% ----------------------------------------------------------
% PLANEJAMENTO E GERENCIAMENTO DO PROJETO
% ----------------------------------------------------------
\chapter{Planejamento e Gerenciamento do Projeto}
Neste capítulo abordaremos a metodologia e ferramenta da gestão da equipe e do projeto, os papéis dos integrantes da equipe e informações a cerca do cronograma sendo seguido no desenvolvimento do projeto e sua documentação.
% ----------------------------------------------------------
% METODOLOGIA DE GESTÃO
% ----------------------------------------------------------
% ----------------------------------------------------------
% GESTÃO E DESENVOLVIMENTO DO PROJETO
% ----------------------------------------------------------
\section{Gestão e Desenvolvimento do Projeto}
A equipe decidiu por utilizar a metodologia ágil \textit{\gls{scrum}}, juntamente com a ferramenta de gerenciamento \gls{jira}. O \textit{\gls{scrum}} possui três fases, uma inicial de planejamento geral, uma intermediária de produção e uma final de encerramento. A fase intermediária se trata de uma série de iterações, onde em cada iteração são desenvolvidas atividades/funcionalidades a serem entregues/incrementadas. Estas iterações são chamadas de \textit{\glspl{sprint}} \cite{sommerville}, cuja duração é fixa e a equipe decidiu por durar uma semana (7 dias) no primeiro semestre e duas semanas (14 dias) no segundo semestre. Todas as atividades, elementos e artefatos que precisaram ser produzidos foram organizados, monitorados e atribuídos aos membros da equipe via \gls{jira}, onde foi possível verificar o status da atividade, assim como marcar prazos.
% ----------------------------------------------------------
% ORGANIZAÇÃO DA EQUIPE
% ----------------------------------------------------------
% ----------------------------------------------------------
% ORGANIZAÇÃO DA EQUIPE
% ----------------------------------------------------------
\section{Organização da equipe}
Após avaliarmos as principais competências de cada integrante da equipe, resolvemos separar as tarefas de cada um como indicado no \autoref{responsabilidades}.

\begin{quadro}[H]
	\caption{Divisão de responsabilidades da equipe.}
	\centering
	\begin{tabular}{| p{0.30\linewidth} | c | c | c | c | c | c | c |}
			\hline
			\thead[l]{Responsabilidade} & \thead{Bruna} & \thead{Daniel} & \thead{Igor} & \thead{Leonardo} & \thead{Lucas} & \thead{Marcelo}\\
			\hline
			Back-End. &  &  & X & X &  & X\\
			\hline
			Front-End. & X & X &  & X & X & \\
			\hline
			Banco de Dados. &  & X & X &  &  & \\
			\hline
			Blog. & X & X & X & X & X & X\\
			\hline
			Documentação. & X & X & X & X & X & X\\
			\hline
			Design. & X &  &  &  & X & \\
			\hline
			Gestão. & X &  &  &  &  & \\
			\hline
			
		\end{tabular}
	\fonte{Os Autores}
	\label{responsabilidades}
\end{quadro}

Considerando os papéis inerentes ao \gls{scrum} e as responsabilidades expostas no \autoref{responsabilidades}, o papel do \gls{scrummaster} será desempenhado pela integrante Bruna da Silva Pires, já a equipe de desenvolvimento será composta por todos os integrantes da equipe, sem exceção.

%doc da tabela: https://docs.google.com/document/d/1fCR9h-WhtyQsylmvLcXadEjcmHYoxb3MX57AzWBtPfs/edit

% ----------------------------------------------------------
% GESTÃO DE TEMPO / CRONOGRAMA
% ----------------------------------------------------------
% ----------------------------------------------------------
% GESTÃO DE TEMPO / CRONOGRAMA
% ----------------------------------------------------------
\section{Cronograma}

No início do projeto tínhamos uma organização dos macro itens (\textit{Epics}) que precisavam ser desenvolvidos com base nas entregas da disciplina de PI1A5 e dentro dos \textit{Epics} estipulamos as tarefas a serem feitas. Por meio do \gls{jira} podemos ter uma visão geral do andamento dos \textit{Epics} e os prazos, além das \textit{\glspl{sprint}} planejadas, realizadas e aquela que está em andamento.

\begin{figure}[H]
	\centering
	\caption{\label{jira-geral}Roteiro Geral de PI1A5}
	\includegraphics[width=0.95\textwidth]{../imagens/cronograma-geral.png}
	\fonte{Os Autores}
\end{figure}

A \autoref{jira-geral} mostra à esquerda a lista dos \textit{Epics} considerados para a construção do sistema \emph{EstagiEI} no início do projeto, começando pelo Desenho da Aplicação, então Prova de Conceito, \ac{mvp}, Testes, Documentação da Entrega Final, Correções e Melhorias e Blog. Na parte superior estão as datas e o período englobado por cada \textit{\gls{sprint}}. As marcações em azul mostram o período de duração de cada \textit{Epic}. A seguir, \autoref{jira-detalhe1} e \autoref{jira-detalhe2} mostram de modo mais próximo a figura anterior para fins de melhor visualização.

\begin{figure}[H]
	\centering
	\caption{\label{jira-detalhe1}Roteiro Geral de PI1A5 - Detalhe Inicial}
	\includegraphics[width=0.95\textwidth]{../imagens/cronograma-detalhe1.png}
	\fonte{Os Autores}
\end{figure}

\begin{figure}[H]
	\centering
	\caption{\label{jira-detalhe2}Roteiro Geral de PI1A5 - Detalhe Final}
	\includegraphics[width=0.95\textwidth]{../imagens/cronograma-detalhe2.png}
	\fonte{Os Autores}
\end{figure}

Apresentamos em \autoref{cronogramasem1} as \textit{\glspl{sprint}} e algumas informações expostas em \autoref{jira-geral}, \autoref{jira-detalhe1} e \autoref{jira-detalhe2} do que foi planejado e realizado em PI1A5.

\begin{quadro}[H]
	\caption{Cronograma de \glspl{sprint} - 1º semestre}
	\centering
	\begin{tabular}{| p{0.17\linewidth}  | c | c | p{0.25\linewidth} | c |}
		\hline
		\thead[l]{Sprint} & \thead{Data Inicial} & \thead{Data Final} & \thead[l]{Descrição} & \thead{Status}\\
		\hline
		Desenho da aplicação 1 & 18/04/22 & 25/04/22 & Elaboração da documentação do Desenho da Aplicação. & Concluída\\
		\hline
		Desenho da aplicação 2 & 25/04/22 & 02/05/22 &  Continuação da elaboração do Desenho da Aplicação. Planejamento para a \ac{poc}. & Concluída\\
		\hline
		POC & 02/05/22 & 09/05/22 & Finalização do Desenho da Aplicação. Início do desenvolvimento dos itens da \ac{poc} & Concluída \\
		\hline
		POC 2 & 09/05/22 & 16/05/22 & Continuação do desenvolvimento dos itens da \ac{poc}. & Concluída\\
		\hline
		MVP & 16/05/22 & 23/05/22 & Aproveitamento do que foi desenvolvido para a \ac{poc} com melhorias e ampliação conforme possível para o \ac{mvp}. & Concluída\\
		\hline
		MVP 2 & 23/05/22 & 30/05/22 & Continuação do trabalho no desenvolvimento do \ac{mvp}. & Concluída\\
		\hline
		Revisões: código e documentos & 30/05/22 & 06/06/22 &  Finalização e revisão tanto do desenvolvimento quanto da documentação. & Concluída\\
		\hline
		Preparação para a Apresentação & 06/06/22 & 13/06/22 &  Organização e planejamento da apresentação do projeto e sua documentação. & Concluída\\
		\hline
		Ajustes finais & 13/06/22 & 20/06/22 &  Ajustes a serem feitos para correção e/ou melhoria do projeto apresentado. & Concluída\\
		\hline
		Ajustes finais 2 & 20/06/22 & 27/06/22 &  Continuação de correções e ajustes para a entrega do projeto no semestre. & Concluída\\
		\hline
		Ajustes finais 3 & 27/06/22 & 04/07/22 &  Finalização dos ajustes finais e correções para a entrega definitiva do projeto no semestre. & Concluída\\
		\hline
		
	\end{tabular}
	\fonte{Os Autores}
	\label{cronogramasem1}
\end{quadro}

Para a continuação do projeto, percebemos que uma mudança de planejamento seria necessária. Assim, modificamos nossos \textit{Epics} a fim de estarem mais alinhados com o projeto em desenvolvimento, possuindo subitens de Funcionalidades que se referem a partes menores do produto em si, as quais por sua vez contém as Histórias de Usuário, que descrevem as ações/funções de cada usuário dentro so sistema.

Como o \gls{jira} não possui uma divisão intermediária entre \textit{Epics} e Histórias de Usuário, a seguir apresentamos alguns esquemas da visão que temos das divisões do projeto:

\begin{figure}[H]
	\centering
	\caption{\label{epic-empresa}\textit{Epic} da Empresa}
	\includegraphics[width=0.95\textwidth]{../imagens/epic-empresa.png}
	\fonte{Os Autores}
\end{figure}

Em \autoref{epic-empresa} está a Área da Empresa, que seria uma grande fatia do projeto, contendo as Funcionalidades pertinentes a entidade Empresa e o que se relaciona com ela, como o Gerenciamento de Vagas, o qual se abre em diversas ações que são as Histórias de Usuário dessa funcionalidade.

\begin{figure}[H]
	\centering
	\caption{\label{epic-estudante}\textit{Epic} do Estudante}
	\includegraphics[width=0.95\textwidth]{../imagens/epic-estudante.png}
	\fonte{Os Autores}
\end{figure}

Do mesmo modo como o \textit{Epic} da Empresa, o \textit{Epic} do Estudante também possui Funcionalidades pertinentes a entidade Estudante e as ações que um usuário deste tipo precisa ter no sistema, como o Gerenciamento de Candidaturas.

Além de reestruturarmos o modo como vemos e trabalhamos as etapas do projeto, também adaptamos as \textit{\glspl{sprint}} para serem de duas semanas ao invés de uma, devido à situação de tempo dos membros da equipe. O \autoref{jira-roteiro-geral-sem2} ilustra a nova organização de \textit{Epics} dividas entre \gls{backend} e \gls{frontend} e as \textit{\glspl{sprint}} de suas semanas.

\begin{figure}[H]
	\centering
	\caption{\label{jira-roteiro-geral-sem2}Roteiro Geral para PI2A6}
	\includegraphics[width=0.95\textwidth]{../imagens/jira-roteiro-geral-sem2.png}
	\fonte{Os Autores}
\end{figure}

Considerando tudo o que já foi apontado, nosso Cronograma (a seguir) se tornou mais genérico, pois a cada \gls{sprint} definimos o que seria feito, portanto apenas há algo definido explicitamente no início e no fim da segunda etapa de desenvolvimento, como a entrega final.

\begin{quadro}[H]
	\caption{Cronograma de \glspl{sprint} - 2º semestre}
	\centering
	\begin{tabular}{| p{1.5cm}  | c | c | p{4.5cm} | c |}
		\hline
		\thead[l]{Sprint} & \thead{Data Inicial} & \thead{Data Final} & \thead[l]{Descrição} & \thead{Status}\\
		\hline
		\textit{Sprint} 1 & 11/08/22 & 25/08/22 & Reorganização da equipe, reestruturação do código e do projeto de modo geral & Concluída\\
		\hline
		\textit{Sprint} 2 & 25/08/22 & 08/09/22 &  Refinamento, ajustes e adaptações do que já foi produzido; adequação da documentação existente às alterações & Concluída\\
		\hline
		\textit{Sprint} 3 & 08/09/22 & 22/09/22 & Desenvolvimento e testes de novas funcionalidades do sistema & Concluída \\
		\hline
		\textit{Sprint} 4 & 22/09/22 & 06/10/22 & Desenvolvimento e testes de novas funcionalidades do sistema & Concluída\\
		\hline
		\textit{Sprint} 5 & 06/10/22 & 20/10/22 & Desenvolvimento e testes de novas funcionalidades do sistema & Concluída\\
		\hline
		\textit{Sprint} 6 & 20/10/22 & 03/11/22 & Desenvolvimento e testes de novas funcionalidades do sistema & Concluída\\
		\hline
		\textit{Sprint} 7 & 03/11/22 & 17/11/22 & Ajustes finais, Entrega e Apresentação da aplicação & Em Progresso\\
		\hline
		\textit{Sprint} 8 & 17/11/22 & 01/12/22 & Correções e ajustes; Entrega final & Não Iniciada\\
		\hline
		
	\end{tabular}
	\fonte{Os Autores}
	\label{cronogramasem2}
\end{quadro}
% ----------------------------------------------------------
% ESTATÍSTICA DOS REPOSITÓRIOS
% ----------------------------------------------------------
% ----------------------------------------------------------
% ESTATÍSTICA DOS REPOSITÓRIOS
% ----------------------------------------------------------
\section{Estatísticas dos repositórios}

\subsection{SVN}
%gls\{svn}


\subsection{GitHub}

% ----------------------------------------------------------
% <CAPÍTULO 3>
% ----------------------------------------------------------
% ----------------------------------------------------------
% PLANEJAMENTO E GERENCIAMENTO DO PROJETO
% ----------------------------------------------------------
\chapter{Planejamento e Gerenciamento do Projeto}
Neste capítulo abordaremos a metodologia e ferramenta da gestão da equipe e do projeto, os papéis dos integrantes da equipe e informações a cerca do cronograma sendo seguido no desenvolvimento do projeto e sua documentação.
% ----------------------------------------------------------
% METODOLOGIA DE GESTÃO
% ----------------------------------------------------------
% ----------------------------------------------------------
% GESTÃO E DESENVOLVIMENTO DO PROJETO
% ----------------------------------------------------------
\section{Gestão e Desenvolvimento do Projeto}
A equipe decidiu por utilizar a metodologia ágil \textit{\gls{scrum}}, juntamente com a ferramenta de gerenciamento \gls{jira}. O \textit{\gls{scrum}} possui três fases, uma inicial de planejamento geral, uma intermediária de produção e uma final de encerramento. A fase intermediária se trata de uma série de iterações, onde em cada iteração são desenvolvidas atividades/funcionalidades a serem entregues/incrementadas. Estas iterações são chamadas de \textit{\glspl{sprint}} \cite{sommerville}, cuja duração é fixa e a equipe decidiu por durar uma semana (7 dias) no primeiro semestre e duas semanas (14 dias) no segundo semestre. Todas as atividades, elementos e artefatos que precisaram ser produzidos foram organizados, monitorados e atribuídos aos membros da equipe via \gls{jira}, onde foi possível verificar o status da atividade, assim como marcar prazos.
% ----------------------------------------------------------
% ORGANIZAÇÃO DA EQUIPE
% ----------------------------------------------------------
% ----------------------------------------------------------
% ORGANIZAÇÃO DA EQUIPE
% ----------------------------------------------------------
\section{Organização da equipe}
Após avaliarmos as principais competências de cada integrante da equipe, resolvemos separar as tarefas de cada um como indicado no \autoref{responsabilidades}.

\begin{quadro}[H]
	\caption{Divisão de responsabilidades da equipe.}
	\centering
	\begin{tabular}{| p{0.30\linewidth} | c | c | c | c | c | c | c |}
			\hline
			\thead[l]{Responsabilidade} & \thead{Bruna} & \thead{Daniel} & \thead{Igor} & \thead{Leonardo} & \thead{Lucas} & \thead{Marcelo}\\
			\hline
			Back-End. &  &  & X & X &  & X\\
			\hline
			Front-End. & X & X &  & X & X & \\
			\hline
			Banco de Dados. &  & X & X &  &  & \\
			\hline
			Blog. & X & X & X & X & X & X\\
			\hline
			Documentação. & X & X & X & X & X & X\\
			\hline
			Design. & X &  &  &  & X & \\
			\hline
			Gestão. & X &  &  &  &  & \\
			\hline
			
		\end{tabular}
	\fonte{Os Autores}
	\label{responsabilidades}
\end{quadro}

Considerando os papéis inerentes ao \gls{scrum} e as responsabilidades expostas no \autoref{responsabilidades}, o papel do \gls{scrummaster} será desempenhado pela integrante Bruna da Silva Pires, já a equipe de desenvolvimento será composta por todos os integrantes da equipe, sem exceção.

%doc da tabela: https://docs.google.com/document/d/1fCR9h-WhtyQsylmvLcXadEjcmHYoxb3MX57AzWBtPfs/edit

% ----------------------------------------------------------
% GESTÃO DE TEMPO / CRONOGRAMA
% ----------------------------------------------------------
% ----------------------------------------------------------
% GESTÃO DE TEMPO / CRONOGRAMA
% ----------------------------------------------------------
\section{Cronograma}

No início do projeto tínhamos uma organização dos macro itens (\textit{Epics}) que precisavam ser desenvolvidos com base nas entregas da disciplina de PI1A5 e dentro dos \textit{Epics} estipulamos as tarefas a serem feitas. Por meio do \gls{jira} podemos ter uma visão geral do andamento dos \textit{Epics} e os prazos, além das \textit{\glspl{sprint}} planejadas, realizadas e aquela que está em andamento.

\begin{figure}[H]
	\centering
	\caption{\label{jira-geral}Roteiro Geral de PI1A5}
	\includegraphics[width=0.95\textwidth]{../imagens/cronograma-geral.png}
	\fonte{Os Autores}
\end{figure}

A \autoref{jira-geral} mostra à esquerda a lista dos \textit{Epics} considerados para a construção do sistema \emph{EstagiEI} no início do projeto, começando pelo Desenho da Aplicação, então Prova de Conceito, \ac{mvp}, Testes, Documentação da Entrega Final, Correções e Melhorias e Blog. Na parte superior estão as datas e o período englobado por cada \textit{\gls{sprint}}. As marcações em azul mostram o período de duração de cada \textit{Epic}. A seguir, \autoref{jira-detalhe1} e \autoref{jira-detalhe2} mostram de modo mais próximo a figura anterior para fins de melhor visualização.

\begin{figure}[H]
	\centering
	\caption{\label{jira-detalhe1}Roteiro Geral de PI1A5 - Detalhe Inicial}
	\includegraphics[width=0.95\textwidth]{../imagens/cronograma-detalhe1.png}
	\fonte{Os Autores}
\end{figure}

\begin{figure}[H]
	\centering
	\caption{\label{jira-detalhe2}Roteiro Geral de PI1A5 - Detalhe Final}
	\includegraphics[width=0.95\textwidth]{../imagens/cronograma-detalhe2.png}
	\fonte{Os Autores}
\end{figure}

Apresentamos em \autoref{cronogramasem1} as \textit{\glspl{sprint}} e algumas informações expostas em \autoref{jira-geral}, \autoref{jira-detalhe1} e \autoref{jira-detalhe2} do que foi planejado e realizado em PI1A5.

\begin{quadro}[H]
	\caption{Cronograma de \glspl{sprint} - 1º semestre}
	\centering
	\begin{tabular}{| p{0.17\linewidth}  | c | c | p{0.25\linewidth} | c |}
		\hline
		\thead[l]{Sprint} & \thead{Data Inicial} & \thead{Data Final} & \thead[l]{Descrição} & \thead{Status}\\
		\hline
		Desenho da aplicação 1 & 18/04/22 & 25/04/22 & Elaboração da documentação do Desenho da Aplicação. & Concluída\\
		\hline
		Desenho da aplicação 2 & 25/04/22 & 02/05/22 &  Continuação da elaboração do Desenho da Aplicação. Planejamento para a \ac{poc}. & Concluída\\
		\hline
		POC & 02/05/22 & 09/05/22 & Finalização do Desenho da Aplicação. Início do desenvolvimento dos itens da \ac{poc} & Concluída \\
		\hline
		POC 2 & 09/05/22 & 16/05/22 & Continuação do desenvolvimento dos itens da \ac{poc}. & Concluída\\
		\hline
		MVP & 16/05/22 & 23/05/22 & Aproveitamento do que foi desenvolvido para a \ac{poc} com melhorias e ampliação conforme possível para o \ac{mvp}. & Concluída\\
		\hline
		MVP 2 & 23/05/22 & 30/05/22 & Continuação do trabalho no desenvolvimento do \ac{mvp}. & Concluída\\
		\hline
		Revisões: código e documentos & 30/05/22 & 06/06/22 &  Finalização e revisão tanto do desenvolvimento quanto da documentação. & Concluída\\
		\hline
		Preparação para a Apresentação & 06/06/22 & 13/06/22 &  Organização e planejamento da apresentação do projeto e sua documentação. & Concluída\\
		\hline
		Ajustes finais & 13/06/22 & 20/06/22 &  Ajustes a serem feitos para correção e/ou melhoria do projeto apresentado. & Concluída\\
		\hline
		Ajustes finais 2 & 20/06/22 & 27/06/22 &  Continuação de correções e ajustes para a entrega do projeto no semestre. & Concluída\\
		\hline
		Ajustes finais 3 & 27/06/22 & 04/07/22 &  Finalização dos ajustes finais e correções para a entrega definitiva do projeto no semestre. & Concluída\\
		\hline
		
	\end{tabular}
	\fonte{Os Autores}
	\label{cronogramasem1}
\end{quadro}

Para a continuação do projeto, percebemos que uma mudança de planejamento seria necessária. Assim, modificamos nossos \textit{Epics} a fim de estarem mais alinhados com o projeto em desenvolvimento, possuindo subitens de Funcionalidades que se referem a partes menores do produto em si, as quais por sua vez contém as Histórias de Usuário, que descrevem as ações/funções de cada usuário dentro so sistema.

Como o \gls{jira} não possui uma divisão intermediária entre \textit{Epics} e Histórias de Usuário, a seguir apresentamos alguns esquemas da visão que temos das divisões do projeto:

\begin{figure}[H]
	\centering
	\caption{\label{epic-empresa}\textit{Epic} da Empresa}
	\includegraphics[width=0.95\textwidth]{../imagens/epic-empresa.png}
	\fonte{Os Autores}
\end{figure}

Em \autoref{epic-empresa} está a Área da Empresa, que seria uma grande fatia do projeto, contendo as Funcionalidades pertinentes a entidade Empresa e o que se relaciona com ela, como o Gerenciamento de Vagas, o qual se abre em diversas ações que são as Histórias de Usuário dessa funcionalidade.

\begin{figure}[H]
	\centering
	\caption{\label{epic-estudante}\textit{Epic} do Estudante}
	\includegraphics[width=0.95\textwidth]{../imagens/epic-estudante.png}
	\fonte{Os Autores}
\end{figure}

Do mesmo modo como o \textit{Epic} da Empresa, o \textit{Epic} do Estudante também possui Funcionalidades pertinentes a entidade Estudante e as ações que um usuário deste tipo precisa ter no sistema, como o Gerenciamento de Candidaturas.

Além de reestruturarmos o modo como vemos e trabalhamos as etapas do projeto, também adaptamos as \textit{\glspl{sprint}} para serem de duas semanas ao invés de uma, devido à situação de tempo dos membros da equipe. O \autoref{jira-roteiro-geral-sem2} ilustra a nova organização de \textit{Epics} dividas entre \gls{backend} e \gls{frontend} e as \textit{\glspl{sprint}} de suas semanas.

\begin{figure}[H]
	\centering
	\caption{\label{jira-roteiro-geral-sem2}Roteiro Geral para PI2A6}
	\includegraphics[width=0.95\textwidth]{../imagens/jira-roteiro-geral-sem2.png}
	\fonte{Os Autores}
\end{figure}

Considerando tudo o que já foi apontado, nosso Cronograma (a seguir) se tornou mais genérico, pois a cada \gls{sprint} definimos o que seria feito, portanto apenas há algo definido explicitamente no início e no fim da segunda etapa de desenvolvimento, como a entrega final.

\begin{quadro}[H]
	\caption{Cronograma de \glspl{sprint} - 2º semestre}
	\centering
	\begin{tabular}{| p{1.5cm}  | c | c | p{4.5cm} | c |}
		\hline
		\thead[l]{Sprint} & \thead{Data Inicial} & \thead{Data Final} & \thead[l]{Descrição} & \thead{Status}\\
		\hline
		\textit{Sprint} 1 & 11/08/22 & 25/08/22 & Reorganização da equipe, reestruturação do código e do projeto de modo geral & Concluída\\
		\hline
		\textit{Sprint} 2 & 25/08/22 & 08/09/22 &  Refinamento, ajustes e adaptações do que já foi produzido; adequação da documentação existente às alterações & Concluída\\
		\hline
		\textit{Sprint} 3 & 08/09/22 & 22/09/22 & Desenvolvimento e testes de novas funcionalidades do sistema & Concluída \\
		\hline
		\textit{Sprint} 4 & 22/09/22 & 06/10/22 & Desenvolvimento e testes de novas funcionalidades do sistema & Concluída\\
		\hline
		\textit{Sprint} 5 & 06/10/22 & 20/10/22 & Desenvolvimento e testes de novas funcionalidades do sistema & Concluída\\
		\hline
		\textit{Sprint} 6 & 20/10/22 & 03/11/22 & Desenvolvimento e testes de novas funcionalidades do sistema & Concluída\\
		\hline
		\textit{Sprint} 7 & 03/11/22 & 17/11/22 & Ajustes finais, Entrega e Apresentação da aplicação & Em Progresso\\
		\hline
		\textit{Sprint} 8 & 17/11/22 & 01/12/22 & Correções e ajustes; Entrega final & Não Iniciada\\
		\hline
		
	\end{tabular}
	\fonte{Os Autores}
	\label{cronogramasem2}
\end{quadro}
% ----------------------------------------------------------
% ESTATÍSTICA DOS REPOSITÓRIOS
% ----------------------------------------------------------
% ----------------------------------------------------------
% ESTATÍSTICA DOS REPOSITÓRIOS
% ----------------------------------------------------------
\section{Estatísticas dos repositórios}

\subsection{SVN}
%gls\{svn}


\subsection{GitHub}

% ----------------------------------------------------------
% <CAPÍTULO 4>
% ----------------------------------------------------------
% ----------------------------------------------------------
% PLANEJAMENTO E GERENCIAMENTO DO PROJETO
% ----------------------------------------------------------
\chapter{Planejamento e Gerenciamento do Projeto}
Neste capítulo abordaremos a metodologia e ferramenta da gestão da equipe e do projeto, os papéis dos integrantes da equipe e informações a cerca do cronograma sendo seguido no desenvolvimento do projeto e sua documentação.
% ----------------------------------------------------------
% METODOLOGIA DE GESTÃO
% ----------------------------------------------------------
% ----------------------------------------------------------
% GESTÃO E DESENVOLVIMENTO DO PROJETO
% ----------------------------------------------------------
\section{Gestão e Desenvolvimento do Projeto}
A equipe decidiu por utilizar a metodologia ágil \textit{\gls{scrum}}, juntamente com a ferramenta de gerenciamento \gls{jira}. O \textit{\gls{scrum}} possui três fases, uma inicial de planejamento geral, uma intermediária de produção e uma final de encerramento. A fase intermediária se trata de uma série de iterações, onde em cada iteração são desenvolvidas atividades/funcionalidades a serem entregues/incrementadas. Estas iterações são chamadas de \textit{\glspl{sprint}} \cite{sommerville}, cuja duração é fixa e a equipe decidiu por durar uma semana (7 dias) no primeiro semestre e duas semanas (14 dias) no segundo semestre. Todas as atividades, elementos e artefatos que precisaram ser produzidos foram organizados, monitorados e atribuídos aos membros da equipe via \gls{jira}, onde foi possível verificar o status da atividade, assim como marcar prazos.
% ----------------------------------------------------------
% ORGANIZAÇÃO DA EQUIPE
% ----------------------------------------------------------
% ----------------------------------------------------------
% ORGANIZAÇÃO DA EQUIPE
% ----------------------------------------------------------
\section{Organização da equipe}
Após avaliarmos as principais competências de cada integrante da equipe, resolvemos separar as tarefas de cada um como indicado no \autoref{responsabilidades}.

\begin{quadro}[H]
	\caption{Divisão de responsabilidades da equipe.}
	\centering
	\begin{tabular}{| p{0.30\linewidth} | c | c | c | c | c | c | c |}
			\hline
			\thead[l]{Responsabilidade} & \thead{Bruna} & \thead{Daniel} & \thead{Igor} & \thead{Leonardo} & \thead{Lucas} & \thead{Marcelo}\\
			\hline
			Back-End. &  &  & X & X &  & X\\
			\hline
			Front-End. & X & X &  & X & X & \\
			\hline
			Banco de Dados. &  & X & X &  &  & \\
			\hline
			Blog. & X & X & X & X & X & X\\
			\hline
			Documentação. & X & X & X & X & X & X\\
			\hline
			Design. & X &  &  &  & X & \\
			\hline
			Gestão. & X &  &  &  &  & \\
			\hline
			
		\end{tabular}
	\fonte{Os Autores}
	\label{responsabilidades}
\end{quadro}

Considerando os papéis inerentes ao \gls{scrum} e as responsabilidades expostas no \autoref{responsabilidades}, o papel do \gls{scrummaster} será desempenhado pela integrante Bruna da Silva Pires, já a equipe de desenvolvimento será composta por todos os integrantes da equipe, sem exceção.

%doc da tabela: https://docs.google.com/document/d/1fCR9h-WhtyQsylmvLcXadEjcmHYoxb3MX57AzWBtPfs/edit

% ----------------------------------------------------------
% GESTÃO DE TEMPO / CRONOGRAMA
% ----------------------------------------------------------
% ----------------------------------------------------------
% GESTÃO DE TEMPO / CRONOGRAMA
% ----------------------------------------------------------
\section{Cronograma}

No início do projeto tínhamos uma organização dos macro itens (\textit{Epics}) que precisavam ser desenvolvidos com base nas entregas da disciplina de PI1A5 e dentro dos \textit{Epics} estipulamos as tarefas a serem feitas. Por meio do \gls{jira} podemos ter uma visão geral do andamento dos \textit{Epics} e os prazos, além das \textit{\glspl{sprint}} planejadas, realizadas e aquela que está em andamento.

\begin{figure}[H]
	\centering
	\caption{\label{jira-geral}Roteiro Geral de PI1A5}
	\includegraphics[width=0.95\textwidth]{../imagens/cronograma-geral.png}
	\fonte{Os Autores}
\end{figure}

A \autoref{jira-geral} mostra à esquerda a lista dos \textit{Epics} considerados para a construção do sistema \emph{EstagiEI} no início do projeto, começando pelo Desenho da Aplicação, então Prova de Conceito, \ac{mvp}, Testes, Documentação da Entrega Final, Correções e Melhorias e Blog. Na parte superior estão as datas e o período englobado por cada \textit{\gls{sprint}}. As marcações em azul mostram o período de duração de cada \textit{Epic}. A seguir, \autoref{jira-detalhe1} e \autoref{jira-detalhe2} mostram de modo mais próximo a figura anterior para fins de melhor visualização.

\begin{figure}[H]
	\centering
	\caption{\label{jira-detalhe1}Roteiro Geral de PI1A5 - Detalhe Inicial}
	\includegraphics[width=0.95\textwidth]{../imagens/cronograma-detalhe1.png}
	\fonte{Os Autores}
\end{figure}

\begin{figure}[H]
	\centering
	\caption{\label{jira-detalhe2}Roteiro Geral de PI1A5 - Detalhe Final}
	\includegraphics[width=0.95\textwidth]{../imagens/cronograma-detalhe2.png}
	\fonte{Os Autores}
\end{figure}

Apresentamos em \autoref{cronogramasem1} as \textit{\glspl{sprint}} e algumas informações expostas em \autoref{jira-geral}, \autoref{jira-detalhe1} e \autoref{jira-detalhe2} do que foi planejado e realizado em PI1A5.

\begin{quadro}[H]
	\caption{Cronograma de \glspl{sprint} - 1º semestre}
	\centering
	\begin{tabular}{| p{0.17\linewidth}  | c | c | p{0.25\linewidth} | c |}
		\hline
		\thead[l]{Sprint} & \thead{Data Inicial} & \thead{Data Final} & \thead[l]{Descrição} & \thead{Status}\\
		\hline
		Desenho da aplicação 1 & 18/04/22 & 25/04/22 & Elaboração da documentação do Desenho da Aplicação. & Concluída\\
		\hline
		Desenho da aplicação 2 & 25/04/22 & 02/05/22 &  Continuação da elaboração do Desenho da Aplicação. Planejamento para a \ac{poc}. & Concluída\\
		\hline
		POC & 02/05/22 & 09/05/22 & Finalização do Desenho da Aplicação. Início do desenvolvimento dos itens da \ac{poc} & Concluída \\
		\hline
		POC 2 & 09/05/22 & 16/05/22 & Continuação do desenvolvimento dos itens da \ac{poc}. & Concluída\\
		\hline
		MVP & 16/05/22 & 23/05/22 & Aproveitamento do que foi desenvolvido para a \ac{poc} com melhorias e ampliação conforme possível para o \ac{mvp}. & Concluída\\
		\hline
		MVP 2 & 23/05/22 & 30/05/22 & Continuação do trabalho no desenvolvimento do \ac{mvp}. & Concluída\\
		\hline
		Revisões: código e documentos & 30/05/22 & 06/06/22 &  Finalização e revisão tanto do desenvolvimento quanto da documentação. & Concluída\\
		\hline
		Preparação para a Apresentação & 06/06/22 & 13/06/22 &  Organização e planejamento da apresentação do projeto e sua documentação. & Concluída\\
		\hline
		Ajustes finais & 13/06/22 & 20/06/22 &  Ajustes a serem feitos para correção e/ou melhoria do projeto apresentado. & Concluída\\
		\hline
		Ajustes finais 2 & 20/06/22 & 27/06/22 &  Continuação de correções e ajustes para a entrega do projeto no semestre. & Concluída\\
		\hline
		Ajustes finais 3 & 27/06/22 & 04/07/22 &  Finalização dos ajustes finais e correções para a entrega definitiva do projeto no semestre. & Concluída\\
		\hline
		
	\end{tabular}
	\fonte{Os Autores}
	\label{cronogramasem1}
\end{quadro}

Para a continuação do projeto, percebemos que uma mudança de planejamento seria necessária. Assim, modificamos nossos \textit{Epics} a fim de estarem mais alinhados com o projeto em desenvolvimento, possuindo subitens de Funcionalidades que se referem a partes menores do produto em si, as quais por sua vez contém as Histórias de Usuário, que descrevem as ações/funções de cada usuário dentro so sistema.

Como o \gls{jira} não possui uma divisão intermediária entre \textit{Epics} e Histórias de Usuário, a seguir apresentamos alguns esquemas da visão que temos das divisões do projeto:

\begin{figure}[H]
	\centering
	\caption{\label{epic-empresa}\textit{Epic} da Empresa}
	\includegraphics[width=0.95\textwidth]{../imagens/epic-empresa.png}
	\fonte{Os Autores}
\end{figure}

Em \autoref{epic-empresa} está a Área da Empresa, que seria uma grande fatia do projeto, contendo as Funcionalidades pertinentes a entidade Empresa e o que se relaciona com ela, como o Gerenciamento de Vagas, o qual se abre em diversas ações que são as Histórias de Usuário dessa funcionalidade.

\begin{figure}[H]
	\centering
	\caption{\label{epic-estudante}\textit{Epic} do Estudante}
	\includegraphics[width=0.95\textwidth]{../imagens/epic-estudante.png}
	\fonte{Os Autores}
\end{figure}

Do mesmo modo como o \textit{Epic} da Empresa, o \textit{Epic} do Estudante também possui Funcionalidades pertinentes a entidade Estudante e as ações que um usuário deste tipo precisa ter no sistema, como o Gerenciamento de Candidaturas.

Além de reestruturarmos o modo como vemos e trabalhamos as etapas do projeto, também adaptamos as \textit{\glspl{sprint}} para serem de duas semanas ao invés de uma, devido à situação de tempo dos membros da equipe. O \autoref{jira-roteiro-geral-sem2} ilustra a nova organização de \textit{Epics} dividas entre \gls{backend} e \gls{frontend} e as \textit{\glspl{sprint}} de suas semanas.

\begin{figure}[H]
	\centering
	\caption{\label{jira-roteiro-geral-sem2}Roteiro Geral para PI2A6}
	\includegraphics[width=0.95\textwidth]{../imagens/jira-roteiro-geral-sem2.png}
	\fonte{Os Autores}
\end{figure}

Considerando tudo o que já foi apontado, nosso Cronograma (a seguir) se tornou mais genérico, pois a cada \gls{sprint} definimos o que seria feito, portanto apenas há algo definido explicitamente no início e no fim da segunda etapa de desenvolvimento, como a entrega final.

\begin{quadro}[H]
	\caption{Cronograma de \glspl{sprint} - 2º semestre}
	\centering
	\begin{tabular}{| p{1.5cm}  | c | c | p{4.5cm} | c |}
		\hline
		\thead[l]{Sprint} & \thead{Data Inicial} & \thead{Data Final} & \thead[l]{Descrição} & \thead{Status}\\
		\hline
		\textit{Sprint} 1 & 11/08/22 & 25/08/22 & Reorganização da equipe, reestruturação do código e do projeto de modo geral & Concluída\\
		\hline
		\textit{Sprint} 2 & 25/08/22 & 08/09/22 &  Refinamento, ajustes e adaptações do que já foi produzido; adequação da documentação existente às alterações & Concluída\\
		\hline
		\textit{Sprint} 3 & 08/09/22 & 22/09/22 & Desenvolvimento e testes de novas funcionalidades do sistema & Concluída \\
		\hline
		\textit{Sprint} 4 & 22/09/22 & 06/10/22 & Desenvolvimento e testes de novas funcionalidades do sistema & Concluída\\
		\hline
		\textit{Sprint} 5 & 06/10/22 & 20/10/22 & Desenvolvimento e testes de novas funcionalidades do sistema & Concluída\\
		\hline
		\textit{Sprint} 6 & 20/10/22 & 03/11/22 & Desenvolvimento e testes de novas funcionalidades do sistema & Concluída\\
		\hline
		\textit{Sprint} 7 & 03/11/22 & 17/11/22 & Ajustes finais, Entrega e Apresentação da aplicação & Em Progresso\\
		\hline
		\textit{Sprint} 8 & 17/11/22 & 01/12/22 & Correções e ajustes; Entrega final & Não Iniciada\\
		\hline
		
	\end{tabular}
	\fonte{Os Autores}
	\label{cronogramasem2}
\end{quadro}
% ----------------------------------------------------------
% ESTATÍSTICA DOS REPOSITÓRIOS
% ----------------------------------------------------------
% ----------------------------------------------------------
% ESTATÍSTICA DOS REPOSITÓRIOS
% ----------------------------------------------------------
\section{Estatísticas dos repositórios}

\subsection{SVN}
%gls\{svn}


\subsection{GitHub}

% ----------------------------------------------------------
% <CAPÍTULO 5>
% ----------------------------------------------------------
% ----------------------------------------------------------
% PLANEJAMENTO E GERENCIAMENTO DO PROJETO
% ----------------------------------------------------------
\chapter{Planejamento e Gerenciamento do Projeto}
Neste capítulo abordaremos a metodologia e ferramenta da gestão da equipe e do projeto, os papéis dos integrantes da equipe e informações a cerca do cronograma sendo seguido no desenvolvimento do projeto e sua documentação.
% ----------------------------------------------------------
% METODOLOGIA DE GESTÃO
% ----------------------------------------------------------
% ----------------------------------------------------------
% GESTÃO E DESENVOLVIMENTO DO PROJETO
% ----------------------------------------------------------
\section{Gestão e Desenvolvimento do Projeto}
A equipe decidiu por utilizar a metodologia ágil \textit{\gls{scrum}}, juntamente com a ferramenta de gerenciamento \gls{jira}. O \textit{\gls{scrum}} possui três fases, uma inicial de planejamento geral, uma intermediária de produção e uma final de encerramento. A fase intermediária se trata de uma série de iterações, onde em cada iteração são desenvolvidas atividades/funcionalidades a serem entregues/incrementadas. Estas iterações são chamadas de \textit{\glspl{sprint}} \cite{sommerville}, cuja duração é fixa e a equipe decidiu por durar uma semana (7 dias) no primeiro semestre e duas semanas (14 dias) no segundo semestre. Todas as atividades, elementos e artefatos que precisaram ser produzidos foram organizados, monitorados e atribuídos aos membros da equipe via \gls{jira}, onde foi possível verificar o status da atividade, assim como marcar prazos.
% ----------------------------------------------------------
% ORGANIZAÇÃO DA EQUIPE
% ----------------------------------------------------------
% ----------------------------------------------------------
% ORGANIZAÇÃO DA EQUIPE
% ----------------------------------------------------------
\section{Organização da equipe}
Após avaliarmos as principais competências de cada integrante da equipe, resolvemos separar as tarefas de cada um como indicado no \autoref{responsabilidades}.

\begin{quadro}[H]
	\caption{Divisão de responsabilidades da equipe.}
	\centering
	\begin{tabular}{| p{0.30\linewidth} | c | c | c | c | c | c | c |}
			\hline
			\thead[l]{Responsabilidade} & \thead{Bruna} & \thead{Daniel} & \thead{Igor} & \thead{Leonardo} & \thead{Lucas} & \thead{Marcelo}\\
			\hline
			Back-End. &  &  & X & X &  & X\\
			\hline
			Front-End. & X & X &  & X & X & \\
			\hline
			Banco de Dados. &  & X & X &  &  & \\
			\hline
			Blog. & X & X & X & X & X & X\\
			\hline
			Documentação. & X & X & X & X & X & X\\
			\hline
			Design. & X &  &  &  & X & \\
			\hline
			Gestão. & X &  &  &  &  & \\
			\hline
			
		\end{tabular}
	\fonte{Os Autores}
	\label{responsabilidades}
\end{quadro}

Considerando os papéis inerentes ao \gls{scrum} e as responsabilidades expostas no \autoref{responsabilidades}, o papel do \gls{scrummaster} será desempenhado pela integrante Bruna da Silva Pires, já a equipe de desenvolvimento será composta por todos os integrantes da equipe, sem exceção.

%doc da tabela: https://docs.google.com/document/d/1fCR9h-WhtyQsylmvLcXadEjcmHYoxb3MX57AzWBtPfs/edit

% ----------------------------------------------------------
% GESTÃO DE TEMPO / CRONOGRAMA
% ----------------------------------------------------------
% ----------------------------------------------------------
% GESTÃO DE TEMPO / CRONOGRAMA
% ----------------------------------------------------------
\section{Cronograma}

No início do projeto tínhamos uma organização dos macro itens (\textit{Epics}) que precisavam ser desenvolvidos com base nas entregas da disciplina de PI1A5 e dentro dos \textit{Epics} estipulamos as tarefas a serem feitas. Por meio do \gls{jira} podemos ter uma visão geral do andamento dos \textit{Epics} e os prazos, além das \textit{\glspl{sprint}} planejadas, realizadas e aquela que está em andamento.

\begin{figure}[H]
	\centering
	\caption{\label{jira-geral}Roteiro Geral de PI1A5}
	\includegraphics[width=0.95\textwidth]{../imagens/cronograma-geral.png}
	\fonte{Os Autores}
\end{figure}

A \autoref{jira-geral} mostra à esquerda a lista dos \textit{Epics} considerados para a construção do sistema \emph{EstagiEI} no início do projeto, começando pelo Desenho da Aplicação, então Prova de Conceito, \ac{mvp}, Testes, Documentação da Entrega Final, Correções e Melhorias e Blog. Na parte superior estão as datas e o período englobado por cada \textit{\gls{sprint}}. As marcações em azul mostram o período de duração de cada \textit{Epic}. A seguir, \autoref{jira-detalhe1} e \autoref{jira-detalhe2} mostram de modo mais próximo a figura anterior para fins de melhor visualização.

\begin{figure}[H]
	\centering
	\caption{\label{jira-detalhe1}Roteiro Geral de PI1A5 - Detalhe Inicial}
	\includegraphics[width=0.95\textwidth]{../imagens/cronograma-detalhe1.png}
	\fonte{Os Autores}
\end{figure}

\begin{figure}[H]
	\centering
	\caption{\label{jira-detalhe2}Roteiro Geral de PI1A5 - Detalhe Final}
	\includegraphics[width=0.95\textwidth]{../imagens/cronograma-detalhe2.png}
	\fonte{Os Autores}
\end{figure}

Apresentamos em \autoref{cronogramasem1} as \textit{\glspl{sprint}} e algumas informações expostas em \autoref{jira-geral}, \autoref{jira-detalhe1} e \autoref{jira-detalhe2} do que foi planejado e realizado em PI1A5.

\begin{quadro}[H]
	\caption{Cronograma de \glspl{sprint} - 1º semestre}
	\centering
	\begin{tabular}{| p{0.17\linewidth}  | c | c | p{0.25\linewidth} | c |}
		\hline
		\thead[l]{Sprint} & \thead{Data Inicial} & \thead{Data Final} & \thead[l]{Descrição} & \thead{Status}\\
		\hline
		Desenho da aplicação 1 & 18/04/22 & 25/04/22 & Elaboração da documentação do Desenho da Aplicação. & Concluída\\
		\hline
		Desenho da aplicação 2 & 25/04/22 & 02/05/22 &  Continuação da elaboração do Desenho da Aplicação. Planejamento para a \ac{poc}. & Concluída\\
		\hline
		POC & 02/05/22 & 09/05/22 & Finalização do Desenho da Aplicação. Início do desenvolvimento dos itens da \ac{poc} & Concluída \\
		\hline
		POC 2 & 09/05/22 & 16/05/22 & Continuação do desenvolvimento dos itens da \ac{poc}. & Concluída\\
		\hline
		MVP & 16/05/22 & 23/05/22 & Aproveitamento do que foi desenvolvido para a \ac{poc} com melhorias e ampliação conforme possível para o \ac{mvp}. & Concluída\\
		\hline
		MVP 2 & 23/05/22 & 30/05/22 & Continuação do trabalho no desenvolvimento do \ac{mvp}. & Concluída\\
		\hline
		Revisões: código e documentos & 30/05/22 & 06/06/22 &  Finalização e revisão tanto do desenvolvimento quanto da documentação. & Concluída\\
		\hline
		Preparação para a Apresentação & 06/06/22 & 13/06/22 &  Organização e planejamento da apresentação do projeto e sua documentação. & Concluída\\
		\hline
		Ajustes finais & 13/06/22 & 20/06/22 &  Ajustes a serem feitos para correção e/ou melhoria do projeto apresentado. & Concluída\\
		\hline
		Ajustes finais 2 & 20/06/22 & 27/06/22 &  Continuação de correções e ajustes para a entrega do projeto no semestre. & Concluída\\
		\hline
		Ajustes finais 3 & 27/06/22 & 04/07/22 &  Finalização dos ajustes finais e correções para a entrega definitiva do projeto no semestre. & Concluída\\
		\hline
		
	\end{tabular}
	\fonte{Os Autores}
	\label{cronogramasem1}
\end{quadro}

Para a continuação do projeto, percebemos que uma mudança de planejamento seria necessária. Assim, modificamos nossos \textit{Epics} a fim de estarem mais alinhados com o projeto em desenvolvimento, possuindo subitens de Funcionalidades que se referem a partes menores do produto em si, as quais por sua vez contém as Histórias de Usuário, que descrevem as ações/funções de cada usuário dentro so sistema.

Como o \gls{jira} não possui uma divisão intermediária entre \textit{Epics} e Histórias de Usuário, a seguir apresentamos alguns esquemas da visão que temos das divisões do projeto:

\begin{figure}[H]
	\centering
	\caption{\label{epic-empresa}\textit{Epic} da Empresa}
	\includegraphics[width=0.95\textwidth]{../imagens/epic-empresa.png}
	\fonte{Os Autores}
\end{figure}

Em \autoref{epic-empresa} está a Área da Empresa, que seria uma grande fatia do projeto, contendo as Funcionalidades pertinentes a entidade Empresa e o que se relaciona com ela, como o Gerenciamento de Vagas, o qual se abre em diversas ações que são as Histórias de Usuário dessa funcionalidade.

\begin{figure}[H]
	\centering
	\caption{\label{epic-estudante}\textit{Epic} do Estudante}
	\includegraphics[width=0.95\textwidth]{../imagens/epic-estudante.png}
	\fonte{Os Autores}
\end{figure}

Do mesmo modo como o \textit{Epic} da Empresa, o \textit{Epic} do Estudante também possui Funcionalidades pertinentes a entidade Estudante e as ações que um usuário deste tipo precisa ter no sistema, como o Gerenciamento de Candidaturas.

Além de reestruturarmos o modo como vemos e trabalhamos as etapas do projeto, também adaptamos as \textit{\glspl{sprint}} para serem de duas semanas ao invés de uma, devido à situação de tempo dos membros da equipe. O \autoref{jira-roteiro-geral-sem2} ilustra a nova organização de \textit{Epics} dividas entre \gls{backend} e \gls{frontend} e as \textit{\glspl{sprint}} de suas semanas.

\begin{figure}[H]
	\centering
	\caption{\label{jira-roteiro-geral-sem2}Roteiro Geral para PI2A6}
	\includegraphics[width=0.95\textwidth]{../imagens/jira-roteiro-geral-sem2.png}
	\fonte{Os Autores}
\end{figure}

Considerando tudo o que já foi apontado, nosso Cronograma (a seguir) se tornou mais genérico, pois a cada \gls{sprint} definimos o que seria feito, portanto apenas há algo definido explicitamente no início e no fim da segunda etapa de desenvolvimento, como a entrega final.

\begin{quadro}[H]
	\caption{Cronograma de \glspl{sprint} - 2º semestre}
	\centering
	\begin{tabular}{| p{1.5cm}  | c | c | p{4.5cm} | c |}
		\hline
		\thead[l]{Sprint} & \thead{Data Inicial} & \thead{Data Final} & \thead[l]{Descrição} & \thead{Status}\\
		\hline
		\textit{Sprint} 1 & 11/08/22 & 25/08/22 & Reorganização da equipe, reestruturação do código e do projeto de modo geral & Concluída\\
		\hline
		\textit{Sprint} 2 & 25/08/22 & 08/09/22 &  Refinamento, ajustes e adaptações do que já foi produzido; adequação da documentação existente às alterações & Concluída\\
		\hline
		\textit{Sprint} 3 & 08/09/22 & 22/09/22 & Desenvolvimento e testes de novas funcionalidades do sistema & Concluída \\
		\hline
		\textit{Sprint} 4 & 22/09/22 & 06/10/22 & Desenvolvimento e testes de novas funcionalidades do sistema & Concluída\\
		\hline
		\textit{Sprint} 5 & 06/10/22 & 20/10/22 & Desenvolvimento e testes de novas funcionalidades do sistema & Concluída\\
		\hline
		\textit{Sprint} 6 & 20/10/22 & 03/11/22 & Desenvolvimento e testes de novas funcionalidades do sistema & Concluída\\
		\hline
		\textit{Sprint} 7 & 03/11/22 & 17/11/22 & Ajustes finais, Entrega e Apresentação da aplicação & Em Progresso\\
		\hline
		\textit{Sprint} 8 & 17/11/22 & 01/12/22 & Correções e ajustes; Entrega final & Não Iniciada\\
		\hline
		
	\end{tabular}
	\fonte{Os Autores}
	\label{cronogramasem2}
\end{quadro}
% ----------------------------------------------------------
% ESTATÍSTICA DOS REPOSITÓRIOS
% ----------------------------------------------------------
% ----------------------------------------------------------
% ESTATÍSTICA DOS REPOSITÓRIOS
% ----------------------------------------------------------
\section{Estatísticas dos repositórios}

\subsection{SVN}
%gls\{svn}


\subsection{GitHub}

% ----------------------------------------------------------
% CONSIDERAÇÕES FINAIS / CONCLUSÃO
% ----------------------------------------------------------
% ----------------------------------------------------------
% PLANEJAMENTO E GERENCIAMENTO DO PROJETO
% ----------------------------------------------------------
\chapter{Planejamento e Gerenciamento do Projeto}
Neste capítulo abordaremos a metodologia e ferramenta da gestão da equipe e do projeto, os papéis dos integrantes da equipe e informações a cerca do cronograma sendo seguido no desenvolvimento do projeto e sua documentação.
% ----------------------------------------------------------
% METODOLOGIA DE GESTÃO
% ----------------------------------------------------------
% ----------------------------------------------------------
% GESTÃO E DESENVOLVIMENTO DO PROJETO
% ----------------------------------------------------------
\section{Gestão e Desenvolvimento do Projeto}
A equipe decidiu por utilizar a metodologia ágil \textit{\gls{scrum}}, juntamente com a ferramenta de gerenciamento \gls{jira}. O \textit{\gls{scrum}} possui três fases, uma inicial de planejamento geral, uma intermediária de produção e uma final de encerramento. A fase intermediária se trata de uma série de iterações, onde em cada iteração são desenvolvidas atividades/funcionalidades a serem entregues/incrementadas. Estas iterações são chamadas de \textit{\glspl{sprint}} \cite{sommerville}, cuja duração é fixa e a equipe decidiu por durar uma semana (7 dias) no primeiro semestre e duas semanas (14 dias) no segundo semestre. Todas as atividades, elementos e artefatos que precisaram ser produzidos foram organizados, monitorados e atribuídos aos membros da equipe via \gls{jira}, onde foi possível verificar o status da atividade, assim como marcar prazos.
% ----------------------------------------------------------
% ORGANIZAÇÃO DA EQUIPE
% ----------------------------------------------------------
% ----------------------------------------------------------
% ORGANIZAÇÃO DA EQUIPE
% ----------------------------------------------------------
\section{Organização da equipe}
Após avaliarmos as principais competências de cada integrante da equipe, resolvemos separar as tarefas de cada um como indicado no \autoref{responsabilidades}.

\begin{quadro}[H]
	\caption{Divisão de responsabilidades da equipe.}
	\centering
	\begin{tabular}{| p{0.30\linewidth} | c | c | c | c | c | c | c |}
			\hline
			\thead[l]{Responsabilidade} & \thead{Bruna} & \thead{Daniel} & \thead{Igor} & \thead{Leonardo} & \thead{Lucas} & \thead{Marcelo}\\
			\hline
			Back-End. &  &  & X & X &  & X\\
			\hline
			Front-End. & X & X &  & X & X & \\
			\hline
			Banco de Dados. &  & X & X &  &  & \\
			\hline
			Blog. & X & X & X & X & X & X\\
			\hline
			Documentação. & X & X & X & X & X & X\\
			\hline
			Design. & X &  &  &  & X & \\
			\hline
			Gestão. & X &  &  &  &  & \\
			\hline
			
		\end{tabular}
	\fonte{Os Autores}
	\label{responsabilidades}
\end{quadro}

Considerando os papéis inerentes ao \gls{scrum} e as responsabilidades expostas no \autoref{responsabilidades}, o papel do \gls{scrummaster} será desempenhado pela integrante Bruna da Silva Pires, já a equipe de desenvolvimento será composta por todos os integrantes da equipe, sem exceção.

%doc da tabela: https://docs.google.com/document/d/1fCR9h-WhtyQsylmvLcXadEjcmHYoxb3MX57AzWBtPfs/edit

% ----------------------------------------------------------
% GESTÃO DE TEMPO / CRONOGRAMA
% ----------------------------------------------------------
% ----------------------------------------------------------
% GESTÃO DE TEMPO / CRONOGRAMA
% ----------------------------------------------------------
\section{Cronograma}

No início do projeto tínhamos uma organização dos macro itens (\textit{Epics}) que precisavam ser desenvolvidos com base nas entregas da disciplina de PI1A5 e dentro dos \textit{Epics} estipulamos as tarefas a serem feitas. Por meio do \gls{jira} podemos ter uma visão geral do andamento dos \textit{Epics} e os prazos, além das \textit{\glspl{sprint}} planejadas, realizadas e aquela que está em andamento.

\begin{figure}[H]
	\centering
	\caption{\label{jira-geral}Roteiro Geral de PI1A5}
	\includegraphics[width=0.95\textwidth]{../imagens/cronograma-geral.png}
	\fonte{Os Autores}
\end{figure}

A \autoref{jira-geral} mostra à esquerda a lista dos \textit{Epics} considerados para a construção do sistema \emph{EstagiEI} no início do projeto, começando pelo Desenho da Aplicação, então Prova de Conceito, \ac{mvp}, Testes, Documentação da Entrega Final, Correções e Melhorias e Blog. Na parte superior estão as datas e o período englobado por cada \textit{\gls{sprint}}. As marcações em azul mostram o período de duração de cada \textit{Epic}. A seguir, \autoref{jira-detalhe1} e \autoref{jira-detalhe2} mostram de modo mais próximo a figura anterior para fins de melhor visualização.

\begin{figure}[H]
	\centering
	\caption{\label{jira-detalhe1}Roteiro Geral de PI1A5 - Detalhe Inicial}
	\includegraphics[width=0.95\textwidth]{../imagens/cronograma-detalhe1.png}
	\fonte{Os Autores}
\end{figure}

\begin{figure}[H]
	\centering
	\caption{\label{jira-detalhe2}Roteiro Geral de PI1A5 - Detalhe Final}
	\includegraphics[width=0.95\textwidth]{../imagens/cronograma-detalhe2.png}
	\fonte{Os Autores}
\end{figure}

Apresentamos em \autoref{cronogramasem1} as \textit{\glspl{sprint}} e algumas informações expostas em \autoref{jira-geral}, \autoref{jira-detalhe1} e \autoref{jira-detalhe2} do que foi planejado e realizado em PI1A5.

\begin{quadro}[H]
	\caption{Cronograma de \glspl{sprint} - 1º semestre}
	\centering
	\begin{tabular}{| p{0.17\linewidth}  | c | c | p{0.25\linewidth} | c |}
		\hline
		\thead[l]{Sprint} & \thead{Data Inicial} & \thead{Data Final} & \thead[l]{Descrição} & \thead{Status}\\
		\hline
		Desenho da aplicação 1 & 18/04/22 & 25/04/22 & Elaboração da documentação do Desenho da Aplicação. & Concluída\\
		\hline
		Desenho da aplicação 2 & 25/04/22 & 02/05/22 &  Continuação da elaboração do Desenho da Aplicação. Planejamento para a \ac{poc}. & Concluída\\
		\hline
		POC & 02/05/22 & 09/05/22 & Finalização do Desenho da Aplicação. Início do desenvolvimento dos itens da \ac{poc} & Concluída \\
		\hline
		POC 2 & 09/05/22 & 16/05/22 & Continuação do desenvolvimento dos itens da \ac{poc}. & Concluída\\
		\hline
		MVP & 16/05/22 & 23/05/22 & Aproveitamento do que foi desenvolvido para a \ac{poc} com melhorias e ampliação conforme possível para o \ac{mvp}. & Concluída\\
		\hline
		MVP 2 & 23/05/22 & 30/05/22 & Continuação do trabalho no desenvolvimento do \ac{mvp}. & Concluída\\
		\hline
		Revisões: código e documentos & 30/05/22 & 06/06/22 &  Finalização e revisão tanto do desenvolvimento quanto da documentação. & Concluída\\
		\hline
		Preparação para a Apresentação & 06/06/22 & 13/06/22 &  Organização e planejamento da apresentação do projeto e sua documentação. & Concluída\\
		\hline
		Ajustes finais & 13/06/22 & 20/06/22 &  Ajustes a serem feitos para correção e/ou melhoria do projeto apresentado. & Concluída\\
		\hline
		Ajustes finais 2 & 20/06/22 & 27/06/22 &  Continuação de correções e ajustes para a entrega do projeto no semestre. & Concluída\\
		\hline
		Ajustes finais 3 & 27/06/22 & 04/07/22 &  Finalização dos ajustes finais e correções para a entrega definitiva do projeto no semestre. & Concluída\\
		\hline
		
	\end{tabular}
	\fonte{Os Autores}
	\label{cronogramasem1}
\end{quadro}

Para a continuação do projeto, percebemos que uma mudança de planejamento seria necessária. Assim, modificamos nossos \textit{Epics} a fim de estarem mais alinhados com o projeto em desenvolvimento, possuindo subitens de Funcionalidades que se referem a partes menores do produto em si, as quais por sua vez contém as Histórias de Usuário, que descrevem as ações/funções de cada usuário dentro so sistema.

Como o \gls{jira} não possui uma divisão intermediária entre \textit{Epics} e Histórias de Usuário, a seguir apresentamos alguns esquemas da visão que temos das divisões do projeto:

\begin{figure}[H]
	\centering
	\caption{\label{epic-empresa}\textit{Epic} da Empresa}
	\includegraphics[width=0.95\textwidth]{../imagens/epic-empresa.png}
	\fonte{Os Autores}
\end{figure}

Em \autoref{epic-empresa} está a Área da Empresa, que seria uma grande fatia do projeto, contendo as Funcionalidades pertinentes a entidade Empresa e o que se relaciona com ela, como o Gerenciamento de Vagas, o qual se abre em diversas ações que são as Histórias de Usuário dessa funcionalidade.

\begin{figure}[H]
	\centering
	\caption{\label{epic-estudante}\textit{Epic} do Estudante}
	\includegraphics[width=0.95\textwidth]{../imagens/epic-estudante.png}
	\fonte{Os Autores}
\end{figure}

Do mesmo modo como o \textit{Epic} da Empresa, o \textit{Epic} do Estudante também possui Funcionalidades pertinentes a entidade Estudante e as ações que um usuário deste tipo precisa ter no sistema, como o Gerenciamento de Candidaturas.

Além de reestruturarmos o modo como vemos e trabalhamos as etapas do projeto, também adaptamos as \textit{\glspl{sprint}} para serem de duas semanas ao invés de uma, devido à situação de tempo dos membros da equipe. O \autoref{jira-roteiro-geral-sem2} ilustra a nova organização de \textit{Epics} dividas entre \gls{backend} e \gls{frontend} e as \textit{\glspl{sprint}} de suas semanas.

\begin{figure}[H]
	\centering
	\caption{\label{jira-roteiro-geral-sem2}Roteiro Geral para PI2A6}
	\includegraphics[width=0.95\textwidth]{../imagens/jira-roteiro-geral-sem2.png}
	\fonte{Os Autores}
\end{figure}

Considerando tudo o que já foi apontado, nosso Cronograma (a seguir) se tornou mais genérico, pois a cada \gls{sprint} definimos o que seria feito, portanto apenas há algo definido explicitamente no início e no fim da segunda etapa de desenvolvimento, como a entrega final.

\begin{quadro}[H]
	\caption{Cronograma de \glspl{sprint} - 2º semestre}
	\centering
	\begin{tabular}{| p{1.5cm}  | c | c | p{4.5cm} | c |}
		\hline
		\thead[l]{Sprint} & \thead{Data Inicial} & \thead{Data Final} & \thead[l]{Descrição} & \thead{Status}\\
		\hline
		\textit{Sprint} 1 & 11/08/22 & 25/08/22 & Reorganização da equipe, reestruturação do código e do projeto de modo geral & Concluída\\
		\hline
		\textit{Sprint} 2 & 25/08/22 & 08/09/22 &  Refinamento, ajustes e adaptações do que já foi produzido; adequação da documentação existente às alterações & Concluída\\
		\hline
		\textit{Sprint} 3 & 08/09/22 & 22/09/22 & Desenvolvimento e testes de novas funcionalidades do sistema & Concluída \\
		\hline
		\textit{Sprint} 4 & 22/09/22 & 06/10/22 & Desenvolvimento e testes de novas funcionalidades do sistema & Concluída\\
		\hline
		\textit{Sprint} 5 & 06/10/22 & 20/10/22 & Desenvolvimento e testes de novas funcionalidades do sistema & Concluída\\
		\hline
		\textit{Sprint} 6 & 20/10/22 & 03/11/22 & Desenvolvimento e testes de novas funcionalidades do sistema & Concluída\\
		\hline
		\textit{Sprint} 7 & 03/11/22 & 17/11/22 & Ajustes finais, Entrega e Apresentação da aplicação & Em Progresso\\
		\hline
		\textit{Sprint} 8 & 17/11/22 & 01/12/22 & Correções e ajustes; Entrega final & Não Iniciada\\
		\hline
		
	\end{tabular}
	\fonte{Os Autores}
	\label{cronogramasem2}
\end{quadro}
% ----------------------------------------------------------
% ESTATÍSTICA DOS REPOSITÓRIOS
% ----------------------------------------------------------
% ----------------------------------------------------------
% ESTATÍSTICA DOS REPOSITÓRIOS
% ----------------------------------------------------------
\section{Estatísticas dos repositórios}

\subsection{SVN}
%gls\{svn}


\subsection{GitHub}



%Teste de citação para gerar referências no modelo.. \citeauthor{SCRUMGUIDE:2013}

% ----------------------------------------------------------
% Referências bibliográficas
% ----------------------------------------------------------
\bibliography{referencias}

\input{pos-glossario.tex}

% ----------------------------------------------------------
% Apêndices
% Documentos gerados pelo próprio autor
% ----------------------------------------------------------

% ---
% Inicia os apêndices
% ---
\begin{apendicesenv}

% Imprime uma página indicando o início dos apêndices
\partapendices

% ----------------------------------------------------------
\chapter{Publicações do Blog}
% ----------------------------------------------------------

\begin{figure}[htb]
	\caption{\label{qr-url-blog}URL do blog da equipe \LaTeX}
	\begin{center}
		\geraQRCode{https://wecodeifsp.blogspot.com/}
		\legend{\url{https://wecodeifsp.blogspot.com/}}
		\fonte{Os Autores.}
	\end{center}
\end{figure}

% ----------------------------------------------------------
%\chapter{POC Overview}
% ----------------------------------------------------------
\includepdf[pages=1, scale=0.7, frame=true, pagecommand=\chapter{POC Overview}\label{poc-overview}]{../artefatos/poc-overview-wecode.pdf}
\includepdf[pages=2, scale=0.7, frame=true, pagecommand={}]{../artefatos/poc-overview-wecode.pdf}

% ----------------------------------------------------------
%\chapter{Tabela de Relação entre Histórias de Usuário, Requisitos e Casos de Uso}
% ----------------------------------------------------------
\includepdf[landscape=true, pages=1, scale=0.7, frame=false, pagecommand=\chapter{Tabela de Relação entre Histórias de Usuário, Requisitos e Casos de Uso}\label{relacao-us-req-uc}]{../artefatos/HistoriasXRequisitosXCasos.pdf}
\includepdf[landscape=true, pages=2-4, scale=0.7, frame=false]{../artefatos/HistoriasXRequisitosXCasos.pdf}

% ----------------------------------------------------------
%\chapter{Plano de Testes}
% ----------------------------------------------------------
\includepdf[landscape=true, pages=1, scale=0.7, frame=false, pagecommand=\chapter{Plano de Testes}\label{plano-de-testes}]{../artefatos/Plano-de-testes.pdf}
\includepdf[landscape=true, pages=2-38, scale=0.7, frame=false]{../artefatos/Plano-de-testes.pdf}

\end{apendicesenv}
% ---


\input{pos-anexos}

\end{document}