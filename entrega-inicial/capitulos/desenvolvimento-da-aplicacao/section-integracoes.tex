\section{Integrações}
Nessa seção apresentamos as integrações elencadas como sendo possíveis e/ou desejáveis para a aplicação, que foram pensadas com base em outras aplicações do mercado.

\subsection{Login com o Google e LinkedIn}
Levando em conta a experiência de usuário, consideramos colocar a opção do estudante se logar através do \ac{sso} dessas empresas. Dessa forma, não seria necessário digitar a senha toda vez que o usuário for usar o sistema, precisando apenas clicar em um botão e fazer o \textit{login} com uma dessas alternativas. Contudo, como as empresas e seus representantes usam um sistema de cadastro e login próprio da nossa aplicação, tornou-se complexo manter mais de uma forma de acesso para o sistema, assim, optamos por não dar continuidade com essa integração.

\subsection{Entrar em contato via \textit{WhatsApp}}
A aplicação tem uma forma da empresa contatar o estudante via \textit{WhatsApp}. Essa integração foi feita via link disponibilizado pela própria empresa que mantém o aplicativo. Dessa forma, com apenas um clique, é possível enviar uma mensagem diretamente ao estudante.

\subsection{Entrar em contato via \textit{e-mail}}
A aplicação permite que a empresa entre em contato com o candidato via \textit{e-mail}, assim facilitando a comunicação entre as partes interessadas.

\subsection{Acessibilidade com VLibras}
A Lei Brasileira de Inclusão, Art. 63, estipula que os sites devem ser acessíveis de modo a garantir o acesso às informações disponíveis \cite{leiinclusao}, assim, realizaremos a integração com a aplicação \gls{vlibras}, que é um tradutor de texto escrito em Português para \ac{libras}. De acordo com o manual do \gls{vlibras}, esta integração foi realizada com a inclusão de um trecho de código na página \ac{html} da aplicação \cite{manualvlibras}.

\subsection{APIs do IBGE e ViaCEP}
Utilizamos a \ac{api} do IBGE para recuperar as localizações das vagas em conjunto com a \ac{api} do ViaCEP a fim de resgatar as informações de endereço dos usuários cadastrados a partir do CEP informado.