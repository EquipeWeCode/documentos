\section{Integrações}
Nessa seção são citadas as possíveis integrações que a aplicação terá, que foram decididas baseadas em outras aplicações do mercado.

\subsection{Login com o Google e LinkedIn}
Levando em conta a experiência de usuário, a aplicação terá a opção do estudante se logar através do \ac{sso} dessas empresas. Dessa forma, não será necessário digitar a senha toda vez que o usuário for usar o sistema, precisando apenas clicar um botão e fazer o \textit{login} em uma dessas alternativas.

Como citado em \autoref{entrega-poc}, a integração com o \textit{login} da empresa \textit{Google} já foi implementado.

\subsection{Entrar em contato via \textit{Whatsapp}}
A aplicação terá, também, uma forma da empresa contatar o estudante via \textit{Whatsapp}. Essa integração será feita via \ac{api} disponibilizada pela própria empresa que mantém o aplicativo. Dessa forma, com apenas um clique, será possível enviar uma mensagem diretamente ao estudante.

\subsection{Acessibilidade com VLibras}
A Lei Brasileira de Inclusão, Art. 63, estipula que os sites devem ser acessíveis de modo a garantir o acesso às informações disponíveis \cite{leiinclusao}, assim, realizaremos a integração com a aplicação \gls{vlibras}, que é um tradutor de texto escrito em Português para \ac{libras}. De acordo com o manual do \gls{vlibras}, esta integração pode ser realizada com a inclusão de um trecho de código na página \ac{html} da aplicação \cite{manualvlibras}.

\subsection{API dos Correios}
Realizaremos uma integração do a \ac{api} dos Correios a fim de resgatar as informações de endereço dos usuários cadastrados a partir do CEP informado.