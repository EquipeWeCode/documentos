% ----------------------------------------------------------
% MANUTENIBILIDADE
% ----------------------------------------------------------
\section{Manutenibilidade}
Para que a aplicação atinja um nível adequado de qualidade é fundamental
que se estabeleça certos requisitos e parâmetros de manutenibilidade, tais como
ferramentas que facilitam esse processo. Através dos critérios estabelecidos, podemos
medir o quanto o processo de desenvolvimento concorda com as boas práticas e incentivar o uso das mesmas.

% ----------------------------------------------------------
% LOGS
% ----------------------------------------------------------
\subsection{\textit{Logs}}
Para o monitoramento da aplicação em tempo de execução, essencialmente na camada de servidor,
os \textit{logs} serão usados para monitorar o estado dos objetos. A ferramenta a ser utilizada
será a implementação de textit{logs} do \textit{\gls{spring boot}} que utiliza a implementação \textit{\gls{logback}}.
A ferramenta permite diversos registros, como:

\begin{itemize}
    \item \textit{debug} 
    \item \textit{info}
    \item \textit{warn} 
    \item \textit{error}
\end{itemize}

Assim, a cada bloco de falha da aplicação um \textit{log} será colocado para que os problemas sejam identificados,
analisados e resolvidos.

% ----------------------------------------------------------
% CODE CONVENTION
% ----------------------------------------------------------
\subsection{\textit{Code Convention}}
Visando facilitar o entendimento mútuo entre a equipe, são feitas as convenções de código com o propósito de padronizar como os integrantes da equipe produzem seus respectivos códigos, de modo que o estilo de programação seja independente de seus autores.
As convenções de código estabelecem estilos para a organização do código textualmente, ou seja, como os comentários são posicionados, nome de variáveis escolhidas.

\subsubsection{Codificação geral}
As convenções adotadas são baseadas na especificação da \gls{sunmicrosystem}, de 1996. É comumente usada no desenvolvimento na linguagem \textit{Java}, e relativamente próxima do padrão adotado no \textit{JavaScript}, podendo destacar os seguintes pontos:

\begin{itemize}
    \item Minimização do uso de variáveis, funções e objetos globais.
    \item Declarações globais estarão de forma preferencial no início do arquivo.
    \item Declaração de variáveis próximo do ponto onde são inicializadas.
    \item Indentação de 4 espaços no\textit{ \gls{backend}} e 2 espaços no \textit{\gls{frontend}}.
    \item Classes e interfaces em \textbf{CamelCase} e substantivos.
    \item Métodos em \textbf{camelCase} e verbos.
    \item Constantes em \textbf{UPPER\_CASE}.
\end{itemize}

No \textit{\gls{backend}} os pacotes são bem divididos, tendo o pacote \textit{entity} para as entidades mapeadas do banco de dados, \textit{controller} para os \textit{controllers} e \textit{\glspl{endpoint}} e \textit{service}, além de outros pacotes para fins de separação de código.

\subsubsection{\textit{Commits}}
Para os \textit{commits} dos repositórios de \textit{\gls{frontend}} e \textit{\gls{backend}} estamos utilizando a convenção de usar prefixos que melhor identificam do que se trata aquele \textit{commit}.

São eles:
\begin{itemize}
	\item \textbf{\textit{fix}:} correção de erros no código;
	\item \textbf{\textit{feat}:} introdução de uma nova funcionalidade;
\end{itemize}

Além disso também realizamos a prática de realizar \textit{Pull Requests} ao invés de mandar as alterações diretamente no ramo principal do repositório.

% ----------------------------------------------------------
% DESIGN PATTERNS 
% ----------------------------------------------------------
\subsection{\textit{Design Patterns} e boas práticas}
Para padrões de projetos, serão essencialmente utilizados 3 padrões muito utilizados pela comunidade de desenvolvimento: \emph{Factory Method}, \emph{Builder} e \emph{Facade}, além da possibilidade de usarmos outros conforme a necessidade.

\subsubsection{\textit{Clean Code}}
O \textit{Clean Code} é um conjunto de boas práticas de programação que visam melhorar o entendimento do código, facilitando a leitura do mesmo. Algumas das prinipais boas práticas são listadas a seguir:

\begin{itemize}
    \item Nomes significativos para as variáveis, classes, métodos, atributos e objetos.
    \item Utilização de constantes e \textit{enums} para evitar números mágicos.
    \item Evitar comentários que são redundantes e podem ser convertidos em códigos.
    \item Utilização de funções pequenas, com uma única responsabilidade abstrata.
    \item Evitar booleanos de forma explícita.
    \item Diminuir a redundância e a repetição de código (\textit{Don't Repeat Yourself}).
    \item Aumentar a ortogonalidade do código: diminuindo as dependências; aumentando o desacoplamento e a independência entre os módulos de modo a deixá-los mais fáceis de serem modificados (\textit{Easy To Change}).
\end{itemize}

\subsubsection{\textit{SOLID}}

O \textit{SOLID} é um acrônimo para 5 (cinco) princípios da programação orientada a objetos, tais princípios são fundamentais para o desenvolvimento e manutenção de software, visto que trazem facilidade e flexibilidade no código em se adequar à mudanças, algo frequente no desenvolvimento. Os princípios do \textit{SOLID} estão expostos no \autoref{solid}.

\begin{quadro}[H]
	\caption{\textit{SOLID Principles}}
	\centering
	\begin{tabular}{| l | p{0.6\linewidth}|}
		\hline
		\thead[l]{Princípio} & \thead[l]{Descrição}\\
		\hline
		\textit{Single Responsiblity Principle} 	& Uma classe deve ter apenas um motivo para mudar.\\
		\hline
		\textit{Open-Closed Principle}				& Uma classe deve estar aberta para extensão e fechada para modificação, recomendando sempre utilizar a herança e não modificar o código-fonte original.\\
		\hline
		\textit{Liskov Substitution Principle}		& Uma classe derivada deve ser substituível por sua classe base.\\
		\hline
		\textit{Interface Segregation Principle}	& Utilizar muitas interfaces específicas é melhor que uma interface genérica.\\
		\hline
		\textit{Dependency Inversion Principle}		& Dependa de abstrações e não de implementações. \\
		\hline
	\end{tabular}
	\fonte{\cite{solidprinciples}}
	\label{solid}
\end{quadro}

\subsubsection{\textit{12 Factor App}}

A aplicação doze-fatores é uma metodologia para construir softwares como serviço que seguem os parâmetros expostos no \autoref{12-factor}.

\begin{quadro}[H]
	\caption{\textit{12 Factor App parameters}}
	\centering
	\begin{tabular}{| l | p{0.6\linewidth}|}
		\hline
		\thead[l]{Parâmetros} & \thead[l]{Descrição}\\
		\hline
		Base de Código 	& Uma base de código com rastreamento utilizando controle de revisão, muitos \textit{\glspl{deploy}}.\\
		\hline
		Dependências	& Declare e isole as dependências.\\
		\hline
		Configurações	& Armazene as configurações no ambiente.\\
		\hline
		Serviços de Apoio	& Trate os serviços de apoio, como recursos ligados.\\
		\hline
		Construa, lance, execute & Separe estritamente os \textit{builds} e execute em estágios.\\
		\hline
		 Processos & Execute a aplicação como um ou mais processos que não armazenam estado.\\
		\hline
		Vínculo de porta & Exporte serviços por ligação de porta.\\
		\hline
		Concorrência & Dimensione por um modelo de processo.\\
		\hline
		Descartabilidade & Maximizar a robustez com inicialização e desligamento rápido.\\
		\hline
		\textit{Dev/prod} semelhantes & Mantenha o desenvolvimento, teste, produção o mais semelhante possível. \\
		\hline
		\textit{Logs} & Trate logs como fluxo de eventos.\\
		\hline
		Processos de \textit{Admin} & Executar tarefas de administração/gerenciamento como processos pontuais.\\
		\hline
	\end{tabular}
	\fonte{\cite{12factor}}
	\label{12-factor}
\end{quadro}

% ----------------------------------------------------------
% INTEGRAÇÃO CONTINUA
% ----------------------------------------------------------
\subsection{Integração continua}
Para manter o serviço sempre atualizado para o usuário, a ferramenta de integração contínua do \gls{herokuci} foi selecionada para a implantação da aplicação no \textit{\gls{backend}} em produção. O mesmo foi feito com o \textit{\gls{netlify}} em relação ao\textit{ \gls{frontend}}.
\subsubsection{\textit{Back-end}}

\begin{enumerate}
	\item Uma mudança é feita no código do \textit{\gls{backend}} e enviada ao repositório no \gls{github};
    \item Após a mudança do código no repositório, uma instância da \gls{herokuci}, identifica automaticamente a linguagem de programação usada; 
    \item O processo do \textit{\gls{deploy}} é iniciado e a \gls{herokuci} constrói o código em uma aplicação temporária;
    \item A aplicação temporária passa por testes paralelos, cujos resultados são mostrados ao usuário através de uma interface;
    \item Caso a aplicação temporária construída passe pelos testes com sucesso, é feito o \textit{\gls{deploy}} da aplicação na plataforma de hospedagem, no caso sendo o \gls{heroku}.
\end{enumerate} 

A implantação dos itens acima se deu por meio de \textit{Actions} do \gls{github}.

\begin{figure}[htb]
	\caption{\label{qr-url-actions-back}URL de \textit{Actions} do \textit{\gls{backend}}}
	\begin{center}
		\geraQRCode{https://github.com/EquipeWeCode/estagiei-backend/blob/develop/.github/workflows/deploy-develop.yml}
		\legend{\url{https://github.com/EquipeWeCode/estagiei-backend/blob/develop/.github/workflows/deploy-develop.yml}}
		\fonte{Os Autores.}
	\end{center}
\end{figure}

\subsubsection{\textit{Front-end}}
\begin{enumerate}
	\item Uma mudança é feita no código do \textit{\gls{frontend}} e enviada ao repositório no \gls{github};
    \item Após uma mudança do código no \gls{github}, uma instância da \textit{\gls{netlify}}, que tem acesso ao código do \gls{github}, identifica automaticamente essa mudança;
    \item O novo código passa por testes, cujos resultados são mostrados ao usuário através de uma interface;
    \item Após passar nos testes com sucesso, é feito o \textit{\gls{deploy}} da aplicação na plataforma de hospedagem, no caso sendo o \textit{\gls{netlify}}.
\end{enumerate} 

A implantação dos itens acima se deu por meio de \textit{Actions} do \gls{github}.
\begin{figure}[htb]
	\caption{\label{qr-url-actions-front}URL de \textit{Actions} do \textit{\gls{frontend}}}
	\begin{center}
		\geraQRCode{https://github.com/EquipeWeCode/estagiei-frontend/blob/develop/.github/workflows/main.yml}
		\legend{\url{https://github.com/EquipeWeCode/estagiei-frontend/blob/develop/.github/workflows/main.yml}}
		\fonte{Os Autores.}
	\end{center}
\end{figure}