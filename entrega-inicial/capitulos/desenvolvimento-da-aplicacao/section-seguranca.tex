% ----------------------------------------------------------
% ARQUITETURA
% ----------------------------------------------------------
\section{Segurança, Privacidade e Legislação}
Para o desenvolvimento da aplicação, foi levado em consideração alguns aspectos de segurança, privacidade e legislação.
A lei brasileira que diz respeito a como lidar com dados de pessoas em plataformas digitais (sobretudo em aplicações disponíveis na internet) é a Nº 13.709 \cite{lgpd},
que está em vigor desde 2020, a \gls{lgpd}.

De acordo com o estabelecido na \gls{lgpd}, a aplicação irá, se necessário, recuperar o mínimo de dados possíveis do usuário para prosseguir com a sua utilização, como \textit{e-mail}, nome
e informações sobre a instituição de ensino do usuário por parte do candidato e o \ac{cnpj} da empresa por parte da empresa que irá cadastrar as vagas. Sempre que for necessário a obtenção de tais informações por parte do sistema, o usuário será alertado de tal ocorrência.

Também foram levadas em consideração algumas outras questões fundamentais de segurança durante o desenvolvimento da aplicação, visto que utilizamos no \textit{\gls{backend}} uma \gls{api} para a transferência de 
dados e comunicação com o \textit{ \gls{frontend}}:
\begin{itemize}
	\item Autenticação e Autorização: As requisições apenas são aceitas se o usuário estiver autenticado no sistema e os \textit{\glspl{endpoint}} funcionam de acordo com a autorização baseada em papeis;
	\item Criptografia: Seguimos o protocolo e padrão \gls{https} para a transferência de mensagens entre \textit{\gls{backend}} e \textit{\gls{frontend}}, de modo que as mensagens fiquem encriptadas e garantir maior segurança na aplicação;
	\item Não exposição de dados sensíveis à aplicação: Durante o desenvolvimento da aplicação, senhas para comunicação com serviços externos e outras ferramentas não ficam expostas em código, e sim, são passadas	através de variáveis de ambiente de modo a não expor chaves e/ou senhas importantes.
	\item Política de senhas: as senhas dos usuários nunca serão armazenadas diretamente no banco de dados; um algoritmo gera um \textit{hash} e faz a sua comparação no momento da autenticação.
\end{itemize}