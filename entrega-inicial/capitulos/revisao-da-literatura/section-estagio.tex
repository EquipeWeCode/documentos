% ----------------------------------------------------------
% ESTÁGIO
% ----------------------------------------------------------
\section{Estágio}
Nesta seção são apresentados os principais elementos do estágio: sua definição, tipos e carga horária seguindo o estabelecido na legislação brasileira.
\subsection{Definição}
De acordo com a lei nº 11.788, de 25 de setembro de 2008, define-se estágio da seguinte forma:
\begin{quoting}[rightmargin=0cm,leftmargin=4cm]
	\begin{SingleSpace}
	{\footnotesize
	Art. 1º  Estágio é ato educativo escolar supervisionado, desenvolvido no ambiente de trabalho, que visa à preparação para o trabalho produtivo de educandos que estejam freqüentando o ensino regular em instituições de educação superior, de educação profissional, de ensino médio, da educação especial e dos anos finais do ensino fundamental, na modalidade profissional da educação de jovens e adultos. \cite{leiestagio}
	}
	\end{SingleSpace}
\end{quoting}

Portanto, o estágio se trata de uma fase intermediária entre os estudos e a entrada no mercado de trabalho, paralelamente ao primeiro, de acordo com as características da modalidade de ensino, projeto pedagógico e situação do estudante.

\subsection{Tipos de estágio}
Os estágios podem ser obrigatórios ou não-obrigatórios, dependendo do que foi previsto no projeto pedagógico do curso no qual o estudante está matriculado. O estágio do tipo obrigatório se caracteriza pelo requisito de cumprimento de uma determinada quantidade de horas estágio, juntamente com a aprovação nas disciplinas do curso, para a obtenção de diploma. O estágio não-obrigatório é opcional e as horas cumpridas são acrescidas à carga horária obrigatória do curso. \cite{leiestagio}

\subsection{Carga horária}
O estágio não é regido pela \ac{clt}, assim possui sua própria especificação de jornada e carga horária. De acordo com o Art. 10 \cite{leiestagio}, a jornada do estágio é definida em um acordo entre a escola e a empresa, ressaltando que não pode ultrapassar:
\begin{quoting}[rightmargin=0cm,leftmargin=4cm]
	\begin{SingleSpace}
	{\footnotesize
	I – 4 (quatro) horas diárias e 20 (vinte) horas semanais, no caso de estudantes de educação especial e dos anos finais do ensino fundamental, na modalidade profissional de educação de jovens e adultos;\\II – 6 (seis) horas diárias e 30 (trinta) horas semanais, no caso de estudantes do ensino superior, da educação profissional de nível médio e do ensino médio regular.\\§ 1°  O estágio relativo a cursos que alternam teoria e prática, nos períodos em que não estão programadas aulas presenciais, poderá ter jornada de até 40 (quarenta) horas semanais, desde que isso esteja previsto no projeto pedagógico do curso e da instituição de ensino. \\§ 2°  Se a instituição de ensino adotar verificações de aprendizagem periódicas ou finais, nos períodos de avaliação, a carga horária do estágio será reduzida pelo menos à metade, segundo estipulado no termo de compromisso, para garantir o bom desempenho do estudante.\cite{leiestagio}
	}
	\end{SingleSpace}
\end{quoting}

Nota-se que a carga horária dos estágios tentam ser de tal forma que o estudante tenha condições mínimas de realizar o estágio e ainda conseguir frequentar as aulas de modo apropriado.