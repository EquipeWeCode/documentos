% ----------------------------------------------------------
% COMPETÊNCIAS
% ----------------------------------------------------------
\section{Competências}

As habilidades que possuímos podem ser classificadas como técnicas e comportamentais, nos termos em inglês, \emph{hard skills} e \emph{soft skills}, respectivamente.

O que chamamos de competências se referem às \emph{soft skills}, ou seja, o conjunto de características comportamentais da pessoa. Essas características podem ser determinantes na busca por estágio e emprego, assim se faz necessário darmos destaque ao entendimento do que são tais habilidades.

\textit{Soft skills} "são capacidades subjetivas, que atuam no espectro comportamental e social do ser humano e não dependem de diplomas ou certificados"\cite{alura_softskills}, ou ainda podem ser compreendidas como "atributos e competências pessoais que permitem ao indivíduo melhorar as suas iterações com os outros e com o mundo em seu redor" \cite{pereira}. As habilidades técnicas são mais fáceis de serem registradas, detectadas, mapeadas e relacionadas, porém o caráter subjetivo das competências faz com que a mesma habilidade possa ser descrita de modos distintos, ainda que similares, como `Trabalho em equipe' pode ser dito como `Saber trabalhar em equipe' ou `Trabalhar bem em equipe', além de comumente serem importantes na obtenção e manutenção de uma vaga de estágio/emprego.

As vagas existentes de diversas áreas buscam por determinados perfis, os quais comumente se referem ao tipo de atitude procurada para preencher a vaga além dos conhecimentos técnicos. Por exemplo, diversas pesquisas indicam que 'Trabalhar em equipe' e 'Proatividade' são \textit{soft skills} bastante requisitadas em funções relacionadas à área de TI \cite{softskillsrequired}. Portanto, conhecer o seu próprio perfil e aprimorar certas características dará maiores chances de encontrar vagas compatíveis consigo.

A equipe cogitou incluir mecanismos de testes que determinassem quais competências os estudantes teriam ou ainda permitir que as empresas criassem seus testes dentro do nosso sistema, contudo, percebemos que tais questões estavam muito fora dos conhecimentos e habilidades dos membros da equipe e, caso executássemos essas ideias, estaríamos arriscando interferir diretamente nas chances de um estudante ser convocado para uma vaga ou teríamos que fiscalizar os testes das empresas de algum modo, este último também indo além do que a equipe poderia ser capaz de executar no momento e num futuro próximo.

Assim, a equipe optou por possibilitar que o próprio estudante indicasse suas competências, como já é feito nos currículos convencionais, porém de modo a serem peças chave no sistema de recomendação das vagas. Na \autoref{softskills} estão listadas as competências parametrizadas para a associação das vagas com os estudantes no sistema \emph{EstagiEI}.