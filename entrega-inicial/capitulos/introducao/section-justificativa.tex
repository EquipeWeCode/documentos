% ----------------------------------------------------------
% JUSTIFICATIVA
% ----------------------------------------------------------
\section{Justificativa} \label{justificativa}
Existe, na contemporaneidade, uma grande dificuldade em adquirir experiência profissional através da prática de estágio, muitas vezes obrigatória nos projetos pedagógicos de cursos profissionalizantes, técnicos, universitários e de outras modalidades de ensino. O estágio
\begin{quoting}[rightmargin=0cm,leftmargin=4cm]
	\begin{SingleSpace}
		{\footnotesize
		[...] é uma etapa fundamental no processo de desenvolvimento e aprendizagem do aluno, porque promove oportunidades de vivenciar na prática conteúdos acadêmicos, propiciando a aquisição de conhecimentos e atitudes relacionadas com a profissão escolhida pelo estagiário. \cite{ciee}
		}
		conteúdo...
	\end{SingleSpace}
\end{quoting}

No entanto, ainda que os grandes objetivos do estágio sejam de aprendizado, aquisição de experiências e um primeiro contato com o mercado de trabalho, existem vagas de estágio com requisitos de experiência prévia, não condizente com a situação do estudante, além disso, também nota-se que existe certa dificuldade de conexão entre a empresa e o candidato, que muitas vezes não obtém o retorno sobre o processo de seleção da vaga.
Tais dificuldades ficam mais perceptíveis nas plataformas que disponibilizam vagas, que não possuem um filtro para não hospedar vagas incoerentes com a condição de estágio nem uma prática de \textit{feedback} ou alerta para os candidatos, que ficam sem saber sobre a situação da vaga nem a sua própria dentro do processo seletivo.