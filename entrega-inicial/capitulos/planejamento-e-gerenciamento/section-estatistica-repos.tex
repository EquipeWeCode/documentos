% ----------------------------------------------------------
% ESTATÍSTICA DOS REPOSITÓRIOS
% ----------------------------------------------------------
\section{Estatísticas dos repositórios}
Nesta seção serão apresentadas as estatísticas de cada versionador de código, com detalhes da atuação de cada integrante da equipe e dos \emph{commits} feitos durante o desenvolvimento.

\subsection{SVN}
Estatísticas sobre o repositório no \gls{svn} foram geradas através do \gls{statsvn} apesar de ter sido usado apenas para atualização recorrente do repositório, então não foram considerados dados estatísticos sobre atividade de cada membro da equipe.

A \autoref{svn-lines-code} mostra a evolução de linhas de código conforme o tempo.
\begin{figure}[H]
	\centering
	\caption{\label{svn-lines-code}Linhas de código - SVN}
	\includegraphics[width=0.95\textwidth]{../imagens/stats/svn-lines-code.png}
	\fonte{Os autores.}
\end{figure}

A \autoref{svn-activity-hour} mostra os horários onde mais foram feitos \emph{commits} em todos os dias.
\begin{figure}[H]
	\centering
	\caption{\label{svn-activity-hour}Atividade por hora do dia - SVN}
	\includegraphics[width=0.95\textwidth]{../imagens/stats/svn-activity-hour.png}
	\fonte{Os autores.}
\end{figure}

A \autoref{svn-activity-day} mostra a quantidade de \emph{commits} por dia da semana, é notório a alta quantidade de atividade aos domingos.
\begin{figure}[H]
	\centering
	\caption{\label{svn-activity-day}Atividade por dia da semana - SVN}
	\includegraphics[width=0.95\textwidth]{../imagens/stats/svn-activity-day.png}
	\fonte{Os autores.}
\end{figure}

\subsection{GitHub}
A partir dos commits feitos nos repositórios, foram levantadas estatísticas sobre o projeto através do \gls{gitstats}, que servem de base para se ter uma ideia de como foi o andamento do projeto e o que cada integrante da equipe fez durante o período.
O \gls{git} foi usado como nosso versionador principal, então os dados abaixo estão de acordo com a atividade de cada integrante.

A \autoref{overview-front} demonstra uma visão geral sobre alguns dados do repositório onde hospedamos o nosso \gls{frontend}.

\begin{figure}[H]
	\centering
	\caption{\label{overview-front}Visão geral - Projeto \gls{frontend}}
	\includegraphics[width=0.95\textwidth]{../imagens/stats/overview-frontend.png}
	\fonte{Os autores}
\end{figure}

A \autoref{overview-back} demonstra uma visão geral sobre alguns dados do repositório onde foi hospedado o backend. É possível notar algumas diferenças entre ele e o \gls{frontend}.

\begin{figure}[htb]
	\caption{\label{qr-url-front}URL do repositório \gls{frontend}}
	\begin{center}
		\geraQRCode{https://github.com/EquipeWeCode/estagiei-frontend/}
		\legend{\url{https://github.com/EquipeWeCode/estagiei-frontend/}}
		\fonte{Os Autores.}
	\end{center}
\end{figure}

\begin{figure}[H]
	\centering
	\caption{\label{overview-back}Visão geral - Projeto \gls{backend}}
	\includegraphics[width=0.95\textwidth]{../imagens/stats/overview-backend.png}
	\fonte{Os autores.}
\end{figure}

\begin{figure}[htb]
	\caption{\label{qr-url-back}URL do repositório \gls{backend}}
	\begin{center}
		\geraQRCode{https://github.com/EquipeWeCode/estagiei-backend}
		\legend{\url{https://github.com/EquipeWeCode/estagiei-backend}}
		\fonte{Os Autores.}
	\end{center}
\end{figure}

A \autoref{overview-doc} demonstra uma visão geral sobre alguns dados principais do repositório onde foi versionado os documentos \LaTeX do projeto.

\begin{figure}[H]
	\centering
	\caption{\label{overview-doc}Visão geral - Projeto Documentos}
	\includegraphics[width=0.95\textwidth]{../imagens/stats/overview-documentos.png}
	\fonte{Os autores.}
\end{figure}

\begin{figure}[htb]
	\caption{\label{qr-url-doc}URL do repositório de documentos \LaTeX}
	\begin{center}
		\geraQRCode{https://github.com/EquipeWeCode/documentos}
		\legend{\url{https://github.com/EquipeWeCode/documentos}}
		\fonte{Os Autores.}
	\end{center}
\end{figure}

A \autoref{lines-of-code-front} demonstra a evolução em número de linhas do repositório onde foi hospedado o \gls{frontend}.
\begin{figure}[H]
	\centering
	\caption{\label{lines-of-code-front}Linhas de código - Projeto \gls{frontend}}
	\includegraphics[width=0.95\textwidth]{../imagens/stats/lines-of-code-frontend.png}
	\fonte{Os autores.}
\end{figure}

A \autoref{lines-of-code-back} demonstra a evolução em número de linhas do repositório onde foi hospedado o \gls{backend}.
\begin{figure}[H]
	\centering
	\caption{\label{lines-of-code-back}Linhas de código - Projeto Backend}
	\includegraphics[width=0.95\textwidth]{../imagens/stats/lines-of-code-backend.png}
	\fonte{Os autores.}
\end{figure}

Cabe ressaltar que o número de linhas não necessariamente diz que foi desenvolvido muito código, como demonstra a \autoref{extensions-frontend}, grande parte das linhas do nosso \gls{frontend} são de arquivos .json, principalmente do package-lock.json, que é gerado automaticamente e documenta e versiona as dependências do projeto.
\begin{figure}[H]
	\centering
	\caption{\label{extensions-frontend}Extensão de arquivos - Projeto \gls{frontend}}
	\includegraphics[width=0.95\textwidth]{../imagens/stats/extensions-frontend.png}
	\fonte{Os autores.}
\end{figure}

A \autoref{list-of-authors-front} lista os autores que contribuíram no repositório \gls{frontend}.
\begin{figure}[H]
	\centering
	\caption{\label{list-of-authors-front}Dias da semana - Projeto \gls{frontend}}
	\includegraphics[width=0.95\textwidth]{../imagens/stats/list-of-authors-frontend.png}
	\fonte{Os autores.}
\end{figure}

A \autoref{list-of-authors-back} lista os autores que contribuíram no repositório \gls{backend}.
\begin{figure}[H]
	\centering
	\caption{\label{list-of-authors-back}Lista de autores - Projeto Backend}
	\includegraphics[width=0.95\textwidth]{../imagens/stats/list-of-authors-backend.png}
	\fonte{Os autores.}
\end{figure}

A \autoref{list-of-authors-doc} lista os autores que contribuíram no repositório de documentos, no qual todos os membros da equipe participaram.
\begin{figure}[H]
	\centering
	\caption{\label{list-of-authors-doc}Lista de autores - Projeto Documentos}
	\includegraphics[width=0.95\textwidth]{../imagens/stats/list-of-authors-documentos.png}
	\fonte{Os autores.}
\end{figure}

A \autoref{days-of-week-frontend} mostra os dias da semana em que foram feitos mais \emph{commits} no repositório \gls{frontend}, é notável que aos domingos a quantidade de \emph{commits} nos fins de semana é maior do que nos dias da semana.
\begin{figure}[H]
	\centering
	\caption{\label{days-of-week-frontend}Dias da semana - Projeto \gls{frontend}}
	\includegraphics[width=0.95\textwidth]{../imagens/stats/days-of-week-frontend.png}
	\fonte{Os autores.}
\end{figure}

A \autoref{days-of-week-backend} mostra os dias da semana em que foram feitos mais \emph{commits} no repositório \gls{backend}, pode-se notar uma alta taxa de \emph{commits} durante o domingo, enquanto o resto da semana permanece de forma quase igual, com exceção da sexta-feira e segunda-feira que possuem um percentual de atividade menor que os demais dias da semana.
\begin{figure}[H]
	\centering
	\caption{\label{days-of-week-backend}Dias da semana - Projeto Backend}
	\includegraphics[width=0.95\textwidth]{../imagens/stats/days-of-week-backend.png}
	\fonte{Os autores.}
\end{figure}