% ----------------------------------------------------------
% ORGANIZAÇÃO DA EQUIPE
% ----------------------------------------------------------
\section{Organização da equipe}
Após avaliarmos as principais competências de cada integrante da equipe, resolvemos separar as tarefas de cada um como indicado no \autoref{responsabilidades}.

\begin{quadro}[H]
	\caption{Divisão de responsabilidades da equipe.}
	\centering
	\begin{tabular}{| p{0.30\linewidth} | c | c | c | c | c | c | c |}
			\hline
			\thead[l]{Responsabilidade} & \thead{Bruna} & \thead{Daniel} & \thead{Igor} & \thead{Leonardo} & \thead{Lucas} & \thead{Marcelo}\\
			\hline
			Back-End. &  &  & x & x &  & x\\
			\hline
			Front-End. & x & x &  & x & x & \\
			\hline
			Banco de Dados. &  & x & x &  &  & \\
			\hline
			Blog. & x & x & x & x & x & x\\
			\hline
			Documentação. & x & x & x & x & x & x\\
			\hline
			Design. &  &  &  &  & x & \\
			\hline
			Gestão. & x &  &  &  &  & \\
			\hline
			
		\end{tabular}
	\fonte{Os Autores}
	\label{responsabilidades}
\end{quadro}

Considerando os papéis inerentes ao \gls{scrum} e as responsabilidades expostas no \autoref{responsabilidades}, o papel do \gls{scrummaster} será desempenhado pela integrante Bruna da Silva Pires, já a equipe de desenvolvimento será composta por todos os integrantes da equipe, sem exceção.

%doc da tabela: https://docs.google.com/document/d/1fCR9h-WhtyQsylmvLcXadEjcmHYoxb3MX57AzWBtPfs/edit
