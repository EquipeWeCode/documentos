% ----------------------------------------------------------
% CONCORRÊNCIA
% ----------------------------------------------------------
\section{Análise de Concorrentes}
Para a elaboração da proposta, foram verificadas algumas soluções já existentes no mercado. A partir disso, as soluções que mais se assemelham com a proposta foram o \textit{Companhia de Estágios}, \textit{Cia de Talentos} e
\textit{Nube}. Com base neste levantamento, podemos observar algumas intersecções de funcionalidades oferecidas. O quadro \ref{concorrentes} permite a melhor visualização deste levantamento.

\begin{quadro}[h]
\caption{Comparação dos aplicativos concorrentes.}
\centering
\ABNTEXfontereduzida
 % \resizebox{\columnwidth}{!}{%
    \begin{tabular}{| p{0.30\linewidth} | c | c | c | c |}
      \hline
      \thead[l]{Funcionalidades} & \thead{Cia de \\Estágios} & \thead{Cia de \\ Talentos} & \thead{Nube} & \thead{Nosso Proj.}\\
      \hline
      Login/Cadastro. & x & x & x & x\\
      \hline
      Aplicar em uma vaga. & x & x & x & x\\
      \hline
      Notificação a cada mudança do status no processo seletivo. &  &  & x & x\\
      \hline
      Recomendação de vagas e/ou empresas aos estudantes de acordo com as suas características. & & & & x\\
      \hline
      Recomendação de estudantes mais compatíveis com as vagas registradas pelas empresas, de acordo com as características da vaga e da empresa. & & & & x\\
      \hline
      Simplificação de contato via \emph{WhatsApp}. & & & & x\\
      \hline
      Denúncias de vagas incoerentes com a realidade. & & & & x\\
      \hline
      \emph{Feedback} de empresas pós-entrevista.  & & & & x\\
      \hline
      
    \end{tabular}
 % }
  \fonte{Os Autores}
  \label{concorrentes}
\end{quadro}