% ----------------------------------------------------------
% ESCOPO
% ----------------------------------------------------------
\section{Escopo}
% ----------------------------------------------------------
% REQUISITOS
% ----------------------------------------------------------
\subsection{Requisitos}

Para o futuro desenvolvimento da aplicação, serão expostos os requisitos funcionais, não-funcionais e regras de negócio que nossa aplicação terá, tais requisitos foram formados a partir de estudos de como irá funcionar os processos de nosso \emph{website}.

% ----------------------------------------------------------
% REQUISITOS FUNCIONAIS
% ----------------------------------------------------------
\subsubsection{Requisitos Funcionais}

Durante nossa análise, decidimos que esses seriam os principais requisitos funcionais do nosso projeto:

\begin{quadro}[H]
\ABNTEXfontereduzida
\label{quadro-Requisitos Funcionais}
\centering
\caption[Requisitos funcionais]{Requisitos funcionais}
    \begin{tabular}{|l|l|}
    \hline
    \multicolumn{1}{|c|}{\textbf{Código}} &
      \multicolumn{1}{c|}{\textbf{Descrição}} \\ \hline
    RF-001 &
      \begin{tabular}[c]{@{}l@{}}Realizar o gerenciamento de vagas entre os candidatos e as empresas de \\ uma forma simplificada\end{tabular} \\ \hline
    RF-002 &
      \begin{tabular}[c]{@{}l@{}}Recomendar vagas para estudantes, empresas para estudantes, estudantes \\ para vagas/empresas\end{tabular} \\ \hline
    RF-003 &
      Manter um histórico de vagas tanto para o candidato, quanto para a empresa \\ \hline
    RF-004 &
      Exibir uma linha do tempo do andamento da vaga \\ \hline
    RF-005 &
      Alertar os estudantes aplicados à vaga sobre cada mudança em seu processo \\ \hline
    RF-006 &
      \begin{tabular}[c]{@{}l@{}}Possibilitar que a empresa possa entrar em contato com os estudantes \\ recomendados/aplicados à vaga\end{tabular} \\ \hline
    RF-007 &
      Possibilitar que a empresa realize mudanças no status de andamento da vaga \\ \hline
    RF-008 &
      \begin{tabular}[c]{@{}l@{}}Possibilitar que o estudante realize um feedback da empresa pós-entrevista, \\ que será visto por outros estudantes\end{tabular} \\ \hline
    RF-009 &
      \begin{tabular}[c]{@{}l@{}}Não permitir o registro de vagas cujas horas de atividades ultrapassem \\ a carga horária prevista por lei de acordo com a situação escolar de cada estudante\end{tabular} \\ \hline
    \end{tabular}
    \fonte{Os Autores}
\end{quadro}

% ----------------------------------------------------------
% REQUISITOS NÃO-FUNCIONAIS
% ----------------------------------------------------------
\subsubsection{Requisitos Não-funcionais}

Os requisitos não-funcionais do nosso projeto estão listados abaixo:
\begin{quadro}[H]
\centering
\ABNTEXfontereduzida
\label{quadro-Requisitos Não-funcionais}
\caption[Requisitos Não-funcionais]{Requisitos Não-Funcionais}
    \begin{tabular}{|l|l|}
    \hline
    \multicolumn{1}{|c|}{\textbf{Código}} & \multicolumn{1}{c|}{\textbf{Descrição}}                                 \\ \hline
    RNF-001                               & O sistema deve oferecer boa usabilidade (Ser fácil de aprender a usar)  \\ \hline
    RNF-002                               & O sistema deve estar disponível 24 horas por dia, 7 dias por semana     \\ \hline
    RNF-003                               & O sistema deve possuir possibilidade de escalabilidade                  \\ \hline
    RNF-004                               & Tempo para o carregamento que satisfaça as expectativas do cliente      \\ \hline
    RNF-005                               & O sistema deve possuir uma taxa de ocorrência de falhas menor que 0.3\% \\ \hline
    RNF-006                               & O sistema deve estar de acordo com a Lei Geral de Proteção de Dados     \\ \hline
    RNF-007 & \begin{tabular}[c]{@{}l@{}}O sistema deve estar de acordo com a lei Nº 11.788, de 25 de setembro de 2008, \\ regulando a carga horária do estágio\end{tabular} \\ \hline
    RNF-008 & \begin{tabular}[c]{@{}l@{}}O sistema deve ser responsivo aos diferentes dispositivos que os usuários \\ podem utilizar para acessá-lo\end{tabular}             \\ \hline
    \end{tabular}
    \fonte{Os Autores}
\end{quadro}

% ----------------------------------------------------------
% REGRAS DE NEGÓCIO
% ----------------------------------------------------------
\subsubsection{Regras de Negócio}
As regras de negócio do nosso projeto estão listados abaixo:

\begin{quadro}[H]
\centering
\ABNTEXfontereduzida
\label{quadro-Regras de negócios}
\caption[Regras de negócios]{Regras de negócios}
    \begin{tabular}{|l|l|l|}
    \hline
    \multicolumn{1}{|c|}{\textbf{Código}} &
      \multicolumn{1}{c|}{\textbf{Descrição}} &
      \multicolumn{1}{c|}{\textbf{Requisito Relacionado}} \\ \hline
    RN-001 &
      \begin{tabular}[c]{@{}l@{}}As vagas a serem cadastradas devem estar \\ coerentes com o perfil buscado\end{tabular} &
      RF-001 \\ \hline
    RN-002 &
      \begin{tabular}[c]{@{}l@{}}Os históricos das vagas devem ser mantido \\ por todo o período\end{tabular} &
      RF-003 \\ \hline
    RN-003 &
      \begin{tabular}[c]{@{}l@{}}A empresa é responsável pelo encaminhamento \\ do status da vaga\end{tabular} &
      RF-007 \\ \hline
    RN-004 &
      \begin{tabular}[c]{@{}l@{}}Para o candidato enviar um feedback, ele deve \\ ter pelo menos iniciado o processo seletivo\end{tabular} &
      RF-008 \\ \hline
    RN-005 &
      \begin{tabular}[c]{@{}l@{}}O feedback pode ser feito de forma anônima, mas o \\ usuário deve estar logado e ter passado pelo processo \\ seletivo\end{tabular} &
      RF-008 \\ \hline
    \end{tabular}
    \fonte{Os Autores}
\end{quadro}