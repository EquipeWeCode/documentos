
% Definições para glossario

% ATENCAO o SHARELATEX GERA O GLOSSARIO/LISTAS DE SIGLAS SOMENTE UMA VEZ
% CASO SEJA FEITA ALGUMA ALTERAÇÃO NA LISTA DE SIGLAS OU GLOSSARIO É NECESSARIO UTILIZAR A OPÇÃO :
% "Clear Cached Files" DISPONIVEL NA VISUALIZAÇÃO DOS LOGS 
% ---
% https://www.sharelatex.com/learn/Glossaries

% Normalmente somente as palavras referenciadas são impressas no glossário, portanto é necessário referenciar utilizando :
% \gls{identificação}            
% \Gls{identificação}            
% \glspl{identificação}            
% \Glspl{identificação}

\newglossaryentry{scrum} {
	name=Scrum,
	description={Metodologia ágil de software concebida por Jeff Sutherland e sua equipe de desenvolvimento no início dos anos 90.}
}

\newglossaryentry{sprint}{
	name={Sprint},
	plural={Sprints},
	description={Unidade de planejamento do Scrum na qual se verifica o trabalho (funcionalidade) a ser entregue, os recursos necessários e ocorre o desenvolvimento do software de fato.} 
}

\newglossaryentry{jira} {
	name=Jira Software,
	description={ Ferramenta de gerenciamento que permite o monitoramento de tarefas e acompanhamento de projetos.}
}
    

\newglossaryentry{linkedin} {
	name=LinkedIn,
	description={Rede social focada em vagas de emprego.}
}   

\newglossaryentry{heroku} {
	name=Heroku,
	description={Plataforma em nuvem como um serviço que suporta diversas linguagens de programação}
} 

\newglossaryentry{linux} {
	name=Linux,
	description={Kernel open-source usado em diversos sistemas operacionais}
}   

\newglossaryentry{aws} {
	name=AWS,
	description={Amazon Web Services - Plataforma em nuvem \emph{on-demand} que disponibiliza diversos serviços web}
}    

\newglossaryentry{vercel} {
	name=Vercel,
	description={Plataforma em nuvem que faz o \emph{host} de páginas web}
}  

\newglossaryentry{rds} {
	name=RDS,
	description={Amazon Relational Databases - Serviço de banco de dados da \gls{aws}}
}  

\newglossaryentry{pai}{
                name={pai},
                plural={pai},
                description={este é uma entrada pai, que possui outras
                subentradas.} }

 \newglossaryentry{componente}{
                name={componente},
                plural={componentes},
                parent=pai,
                description={descriação da entrada componente.} }
 
 \newglossaryentry{filho}{
                name={filho},
                plural={filhos},
                parent=pai,
                description={isto é uma entrada filha da entrada de nome
                \gls{pai}. Trata-se de uma entrada irmã da entrada
                \gls{componente}.} }



\newglossaryentry{crud} {
    name=CRUD,
    plural= {CRUDs},
    description={Create Retrieve Update Delete - Interface de usuário para manutenção em banco de dados que consiste somente nas operações básicas do banco de dados, sem nenhum tipo de inteligencia adicional de forma a facilitar ao usuário com o tratamento de algum processo}
}

\newglossaryentry{jpg} {
    name=JPG,
    description={Imagem em formato JPEG}
}

\newglossaryentry{moodle} {
    name=Moodle,
    description={Plataforma para auxiliar na organização de disciplinas de forma online, no IFSP SPO disponível no endereço \url{https://eadcampus.spo.ifsp.edu.br}}
}


\newglossaryentry{tag} {
    name=tag,
    plural= {tags},
    description={Um marcador, a palavra em inglês significa etiqueta}
}
                
\newglossaryentry{telegram} {
    name=Telegram,
    description={Aplicativo de mensagens instantâneas que também permite mensagens de texto, voz e conferencias de voz e vídeo}
}
